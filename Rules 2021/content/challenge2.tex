\section{Challenge 2: Passing Challenge}
\label{sec:PassingChallenge}

\subsection{Challenge description}
In recent years teamwork has played an increasingly important role in the RoboCup SPL. A fundamental point in the development of this kind of strategy is the passing of the ball between teammates. The goal of this challenge is to push teams to develop an efficient system for the realization of an offensive strategy based on one or more passes between teammates. 

The setting of the challenge is so defined: there are two attackers robots, and two static defenders. The purpose of the attackers' robots is to execute the highest number of passes between them in a limited amount of time. The timeout is set to 5 minutes.
The referee will visualize the challenge execution using video streaming. 
Each team has three attempt on this challenge; the final score for every team is the one obtained in the best of the three attempts. Teams cannot change code between attempts.
\begin{figure}[ht]
\includegraphics[width=0.95\linewidth]{figs/ch_2_full.jpg}
\caption{Full-size field setup. The red zone represents the movement areas of the attacking robots. The blue zone represents the defenders area. }
\label{ch2:zone96}
\centering
\end{figure}

\begin{figure}[ht]
\includegraphics[width=0.95\linewidth]{figs/ch_2_reduced.jpg}
\caption{3/4 size field setup. The red zone represents the movement areas of the attacking robots. The blue zone represents the defenders area.}
\label{ch2:zone64}
\centering
\end{figure}


\subsection{Field Setup}
This challenge is designed to be conducted independently within the individual labs of participating teams. The requirements are the presence of a half full-size field or a whole 3/4 size field. 
The division of the zones of interest of the challenge for the full-size field is shown in Figure \ref{ch2:zone96}. In Figure \ref{ch2:zone64}, is instead shown the division of the zones for the 3/4 size field.
The red zones represent the movement zones available to attacker robots. The blue zone represents the zone of placement of the defender robots.
Field areas should be made visible by demarcating them on the field through the use of tape of any color except white. Anyway, the lines must still be easily visible by the referee in streaming video.


\subsection{Robots positioning}
In order to participate each team has to provide four robots, two active ones and two inactive ones. The two active robots are the attacking robots. The defending robots may be inactive, they can be placed in standing position or crouched, but not laying down.
Defenders must be turned off or otherwise placed out of the robots' communication network (for example, by setting them up as members of an opposing team).
The starting robot's positions are identified by the points ${A,B,C,D,E,F,G,H,I}$ in Figures \ref{ch2:zone96} and \ref{ch2:zone64}.
The points positions are so defined:
\\
\\
\textbf{Full size field -} the reference frame of each point is put on the lower left corner of the corresponding area.
\\
$A = G = (0.6m, 1.8m)$
\\
$B = H = (1.2m, 1.2m)$
\\
$C = I = (1.8m, 0.6m)$
\\
$D = (0.6, 0.55m)$
\\
$E = (1.2m, 0.55m)$
\\
$F = (1.8m, 0.55m)$
\\
\\
\textbf{3/4 size field -} the reference frame of each point is put on the lower left corner of the corresponding area.
\\
$A = G = (0.6m, 0.6m)$
\\
$B = H = (1.2m, 1.2m)$
\\
$C = I = (1.8m, 1.8m)$
\\
$D = (0.6, 0.55m)$
\\
$E = (1.2m, 0.55m)$
\\
$F = (1.8m, 0.55m)$
\\
\\
The attacking robots are placed in two random points between the positions A,B,C, for one robot and G,H,I, for the other one.  Randomly one of the two robots is selected as the initial bearer of the ball. The defenders are positioned randomly in two points selected between the D,E,F positions.
The points disposition is shown in Figure \ref{ch2:zone96} and Figure \ref{ch2:zone64}.

\subsection{Passing definition}
For the correct evaluation of this challenge, it is essential to define the intended passing objectively. It is based on two fundamental requirements:
\begin{itemize}
    \item The ball has been touched by two robots of the same team
    \item The ball moved more than one meter from its starting position
\end{itemize}


\subsection{Episode definition}
Each episode starts with the referee's whistle. An episode can be concluded in four ways: 
\begin{itemize}
    \item[1] The timeout is reached
    \item[2] The defenders touch the ball
    \item[3] One attacker player stays out of its starting red area for more than 30 seconds.

\end{itemize}

\subsection{Points assignment}
The attacking team is awarded one point for each pass achieved (see rules above). If the ball ends out of the boundaries of the red area of the receiving robot after the pass no point is awarded.  

\subsection{Additional Rules}
\begin{itemize}
    \item The attackers can leave the respective red rectangle zone for a maximum time of 30 seconds (in order to retrieve the ball and pass it back), then they have to come back inside the area.
    \item If during a pass the ball touches the defender, even if the ball doesn't stop, the challenge episode is however considered as finished
    \item If the ball ends out of the red areas, it is manually put back in a random point inside the closest red area by the referee.
    \item Points are counted by the referee
    
\end{itemize}
