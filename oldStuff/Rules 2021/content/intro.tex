\section{Introduction}

In 2021, RoboCup will be held virtually around the world, owing to the ongoing COVID 19 situation. The Standard Platform League will take on a new form that is designed to provide all teams an opportunity to competitively measure their skills in robot soccer within the current circumstances. 
%No travel to competition sites or other teams venues will be necessary for the 2021 competition and associated local competitions. 
Live streaming of videos will be incorporated into the event to keep the contact between teams alive as well as presenting an entertaining experience for the public and sponsors.

For 2021, the RoboCup SPL competition will consist of four challenges of increasing complexity:

\begin{enumerate}
    \item Testing game play skills of an individual robot. 
    \item Testing team coordination.
    \item Testing remote competitive game play between different teams. 
    \item Testing the quality of remote deployment using autonomous calibration.
\end{enumerate}

Conditions of the home venues across all SPL teams tend to vary. Thus, these challenges allow teams to participate in some or all of the challenges depending on their capabilities. The first two challenges allow teams to compete at their home venue exclusively using their own robots. The latter two challenges ask teams to compete by remotely deploying their code onto robots at a neutral host venue, which is an important step towards facilitating fully remote soccer games.

Teams participating in the remote challenges will also be expected to participate as neutral host venues so that robot usage is fairly distributed among teams. To facilitate this, the challenge rules and the guidance for referees have been designed to ensure the safe operation and protection of robot hardware in all venues.

The past years have shown that SPL teams are great at finding a balance between competing with other teams to win awards and maintaining the research focus on RoboCup in advancing the field of computing science through publications. Given the restrictions on any event in 2021 we ask all teams to do their best to make this year's SPL successful and enjoyable for everyone. 

\subsection{General Setup for all Challenges}
\label{sec:GeneralSetupChallenges}

The four challenges have been designed for the following general setup:
\begin{enumerate}
    \item An SPL field of at least 6 x 4 m size (outer field lines), with 2 goals and a green astro-turf or carpet surface. See~\ref{sec:field_dim}.
    \item For the passing and and 1 vs 1 challenges, a WiFi configuration similar to a typical SPL competition plus GameController. 
    \item Additional network configuration for venues hosting the 1 vs 1 challenge to allow for remote code setup. See~\ref{sec:arean-org-setup}.
    \item A rules focus on the safe operation and protection of robot hardware. Referees should pay special attention to the safe operation of the NAO's.
    \item Each challenge has as a result a ranked list of all participating teams in the respective challenge.
\end{enumerate}

If teams are unable to support the requirements for a particular challenge, but still wish to participate, then the team should contact the TC to discuss potential options.

\subsection{Overall Ranking}
RoboCup 2021 awards a trophy for winning the overall ranking of the four challenges.

As described in Section~\ref{sec:GeneralSetupChallenges} each challenges gives a final ranking of all participating teams in the respective challenge. Not participating teams are ranked last (If 16 teams participate, these teams will be ranked on position 16).

To evaluate the overall ranking the mean of the positions is calculated for each team over all challenges and the team with the lowest mean, gets the trophy awarded.

\subsection{Code of Honour}

The TC trust all teams to act fair during competition. Thus, we ask all participants to acknowledge a Code of Honour.
\begin{enumerate}
    \item The safe operation of the robots is paramount, especially where teams deploy remote their software onto robots owned by other teams. Referees and participants should act to prevent damage to robots, and be proactive in enforcing rules  designed to minimise damage.
    \item Teams should endeavour to install bug-free and conservative software for remote participation. While remote challenges feature elements of speed, there are strict consequences for damaging robots, and we hope no team will be prevented from competing due to these reasons.
    \item Teams and participants should be open and honest in the presentation of their local challenge entries and capability of their software.  A live stream should be used where possible, and pre-recorded videos uploaded as close as possible to the scheduled completion of the challenge. The use of features indicating date and time is encouraged.
    \item Teams (and host venues) should present a clear view of the field and surrounds. Teams should not use additional sensing equipment or remote control stations.
\end{enumerate} 

It is our experience that SPL teams are extremely gracious and congenial, and we expect that the Code of Honour will be honoured. However, violation of this code may result in immediate disqualification.

\subsection{Registration}

Details on registration procedures will be published and updated here, once a time frame is confirmed by the RoboCup Federation.
%  TODO: not phrase this not use the SPL webpage to not have always to change the rule book? 

\subsection{Key dates}

\begin{itemize}[leftmargin=*,labelsep=0.7cm, labelindent=2cm]
    \item [2021-01-31] Publishing the first draft or rules with mandatory files.
    \item [2021-05-01] Commitment to participate
    \item [2021-05-01] Rules 2021 compliant game controller release available
    \item [2021-05-01] Jersey approval for new or modified jerseys
    \item [2021-06-01] Feedback for teams with fields smaller than 3/4
    \item [2021-06-01] Submission of credentials for arena access to teams
    \item [2021-06-01] Arena network access responsible contact person
    \item [2021-06-08] Testing hours for arena access with remote teams
    \item [2021-06-15] Field\_dimension.json
    \item [2021-06-22] Start of RoboCup 2021
\end{itemize}
