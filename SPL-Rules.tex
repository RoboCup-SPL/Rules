\documentclass[12pt]{article}

\usepackage{times,fullpage,xspace,fancyhdr}
\usepackage[pdftex]{graphicx}
\usepackage[pdftex,
            a4paper,
            colorlinks=true,
            urlcolor=black,
            linkcolor=black,
            citecolor=black,
            %pagebackref,
            %backref,
            bookmarksopen=false,
            bookmarksnumbered=true,
            pdfstartview=FitH]{hyperref}

\usepackage{graphicx}
\usepackage{xspace,color}
\pdfcompresslevel=9
\newcommand{\leaguename}{RoboCup Standard Platform League (Nao) }
\hypersetup{
 pdftitle={\leaguename Rule Book},
 pdfauthor={Technical Committee},
}
\usepackage[latin1]{inputenc}
\usepackage{amsmath}

\newcommand{\ie}{\mbox{i.\,e.}\xspace}
\newcommand{\eg}{\mbox{e.\,g.}\xspace}
\newcommand{\cf}{\mbox{cf.}\xspace}
\newcommand{\comment}[1]{\marginpar{\pdfannot width 4in height .5in depth 8pt {/Subtype /Text /Contents (#1)}}}
\newcommand{\inparagraph}[1]{\paragraph{#1\hspace{-1em} }}

\title{\leaguename Rule Book}
\author{RoboCup Technical Committee}
\date{(2011 rules, as of \today)}

\setlength{\parindent}{0pt}
\setlength{\parskip}{12pt plus 6pt minus 3 pt}
\setcounter{tocdepth}{1}
\widowpenalty=10000
\clubpenalty=10000

\pagestyle{fancy}
\lhead{}
\chead{}
\rhead{}
\lfoot{}
\cfoot{}
\rfoot{}

\renewcommand{\headrulewidth}{0.4pt}
\renewcommand{\footrulewidth}{0.4pt}

\newcommand{\TotalWidth}{5.4~m\xspace}
\newcommand{\TotalLength}{7.4~m\xspace }
\newcommand{\KickOffAutoTime}{45 seconds\xspace}

\begin{document}

\maketitle

\vfill

\tableofcontents
\setcounter{tocdepth}{3}

\thispagestyle{fancy}

\clearpage

\cfoot{\thepage}
\setcounter{page}{1}

\section{Setup of the Environment}

\subsection{Field Construction}
\label{sec:field_dim}

The soccer field is built on a total carpet area of length \TotalLength and width \TotalWidth. The dimensions of the soccer field are shown in Figure~\ref{fig:field_dim}. The construction of the goals is depicted in Figure \ref{fig:goal_dimensions}.

\begin{figure}[b!]
\centerline{\includegraphics[width=\columnwidth]{figs/fieldDimensions2010.pdf}}
\caption{Scale diagram of entire field (dimensions in mm).} \label{fig:field_dim}
\end{figure}

\begin{figure}[htp]
\begin{center}
\leavevmode
\includegraphics[width=0.8\columnwidth]{figs/goals2010_2.png}\\
%\vspace{10pt}
\includegraphics[width=0.8\columnwidth]{figs/goal_with_dims_top.png}
\caption{Dimensions of the goal, viewed from above and from the side}
\label{fig:goal_dimensions}
\end{center}
\end{figure}

\subsection{Lines}\label{sec:field_lines}

All robot-visible lines on the soccer field (side lines, end lines, halfway line, centre circle, corner arcs, and the lines surrounding the penalty areas) are 50~mm in width. The center circle has an outside diameter of 1250~mm.

In addition to the visible lines, there are two invisible lines only relevant to the throw-in procedure (see Section \ref{sec:throw_in}), but not relevant to the construction of the field. These two throw-in lines are 400~mm away from the outside lines and run parallel to them inside the playing area. They are centered lengthwise with respect to the field. Each throw-in line is 4~m long.

\subsection{Field Colors}

The colors of the soccer field are shown in Figure~\ref{fig:field_color}. All items on the RoboCup field are color-coded:

\begin{itemize}

\item The field (carpet) itself is green (color is not specified, but it should not be too dark).

\item The lines on the field are white.

\item The red team defends the yellow goal.

\item The blue team defends the sky-blue goal.

\item Goals~(\cf Figure~\ref{fig:goal_colors}). The posts and top cross bar are either yellow or sky-blue. The support triangles on back of the posts and the net are white (yellow: RAL 1018, sky-blue: RAL 5015, a brighter blue is even better).

\end{itemize}

\begin{figure}[htp]
\begin{center}
\leavevmode
\includegraphics[width=0.5\columnwidth]{figs/goal_2010_back.png}
\includegraphics[width=0.4\columnwidth]{figs/goal_2010_front.png}
\caption{Appearance of the yellow goal. Notice that the support triangles are white.}
\label{fig:goal_colors}
\end{center}
\end{figure}

\begin{figure}[t]
\centerline{\includegraphics[width=\columnwidth]{figs/emptyfield_2010.png}}
\caption{Field colors and layout.}
\label{fig:field_color}
\end{figure}

\subsection{Lighting Conditions}

The lighting conditions depend on the actual competition site. Only ceiling lights may be used.

\subsection{Ball}
\label{sec:ball}

The official ball will be a Mylec orange street hockey ball. They are 65 mm in diameter and weigh 55 grams. These are the \emph{shiny} balls and are \emph{not} gel-filled. These balls are available at a few locations:
\begin{itemize}
\item \href{http://www.amazon.com/Mylec-Weather-Bounce-Hockey-Orange/dp/B002LBDA30/ref=sr\_1\_18?ie=UTF8\&s=sporting-goods\&qid=1259775748\&sr=8-18}{http://www.amazon.com/Mylec-Weather-Bounce-Hockey-Orange/dp/B002LBDA30/ref=sr\_1\_18?ie=UTF8\&s=sporting-goods\&qid=1259775748\&sr=8-18}
%\item \href{http://www.hockeymonkey.com/hockey-a-r-balls-accessorie.html}{http://www.hockeymonkey.com/hockey-a-r-balls-accessorie.html}
\item \href{http://www.puckshop.com/mylec-orange-warm-weather-ball.html}{http://www.puckshop.com/mylec-orange-warm-weather-ball.html}
\end{itemize}


\section{Robot Players}

\subsection{Hardware}

All teams must use Nao humanoid robots manufactured by Aldebaran Robotics. Absolutely no modifications or additions to the robot hardware are allowed. No additional hardware is permitted including off-board sensing or processing systems. Additional sensors besides those originally installed on the robots are likewise not allowed. The only exceptions are:

\begin{itemize}

\item Attaching the jersey numbers provided by the league to the robots.

\item Adding black and white sponsor or team logos to the upper legs of the robots (\cf Figure~\ref{fig:sponsor}). These logos must be at least 50\% white by area.

\item Adding small black and white stickers to the torso of the robots stating the name of the robot, the name of the team, or similar information. These stickers must be at least 50\% white by area.

\item Setting the passive wrist joints to a fixed position either with glue or a transparent or white duct tape.

\item Protecting the fingers with transparent or white duct tape.

\item Use of alternate memory sticks in replacement of the Aldebaran supplied memory sticks.

\end{itemize}

A computer will be provided by the event organizers for the purpose of sending GameController messages to the robots.

\subsection{Goal Keeper}
\label{sec:goal_keeper}

The goal keeper is the only player that is allowed to stay within the penalty area of its own team and to touch the ball with its arms/hands whilst within the penalty area. It always has the jersey number ``1''.

\subsection{Field Players}
\label{sec:field_players}

The field players are not allowed to enter their own penalty area. The three field players robots have the jersey numbers ``2'', ``3'' and ``4''.

\subsection{Team Markers}
\label{sec:team_markers}

Robots will use colored waistbands as team markers. Official waistbands are blue and pink and have been distributed to each team. An example of how the team markers are worn is shown in Figure~\ref{fig:nao_markers}.

\begin{figure}[b]
  \centerline{\begin{tabular}{lll}
      a) & b) & c) \\
      \includegraphics[height=0.28\columnwidth]{figs/pink_front.png}&
      \includegraphics[height=0.28\columnwidth]{figs/blue_angle.png} &
      \includegraphics[height=0.28\columnwidth]{figs/blue_back.png} 
  \end{tabular} }
  \caption{Nao team markers. a) Front view. b) Side view. c) Back view.} \label{fig:nao_markers}
\end{figure}

\begin{figure}[b]
\centerline{\begin{tabular}{ll}
\includegraphics[height=0.35\columnwidth]{figs/naosim_with_logo.png}&
\includegraphics[height=0.35\columnwidth]{figs/naosim_legs_with_logo_closeup.png}
\end{tabular}}
\caption{Example Sponsor/Team Logo placement.}
\label{fig:sponsor}
\end{figure}

\subsection{Communications}

The robots should play without human control. Communication is only allowed among robots on the field and between the robots and the GameController.

\subsubsection{Acoustic Communications}

There are no restrictions on communication between the robots using a microphone or a speaker.

\subsubsection{Wireless Communications}

The only wireless hardware allowed to be used by the teams are the wireless network cards built into the Naos, and the access points provided by the event organizers. All other wireless hardware must be deactivated. A team may be disqualified if one of the team members violates this rule. The MAC-addresses of all Naos participating in the competition will be registered. Only these MAC-addresses can be reached through the access points provided by the event organizers. In addition, the access points will be secured by different SSIDs and WEP keys. Two of the access points will be connected to PCs running the GameController. A third access point is used for practice. It is connected to a switch with one port for each team. Teams must bring their own Ethernet cables.

Each team will get a range of IP addresses that can be used both for their robots and their computers. The IP addresses, channels, SSIDs, and WEP keys of the fields will be announced at the competition site.

Teams can use a bandwidth of up to 500 Kbps of the wireless. This includes any data transferred, namely the actual payload and any protocol overhead created, \eg, by TCP, UDP, or the GameController.

Any form of wireless robot-to-robot communication is allowed, as long as it uses the access points provided by the event organizers (using the so-called ad-hoc mode is prohibited), it does not conflict with TCP/IP or UDP, and the maximum bandwidth allowed for each team is not exceeded. Each team will be assigned a range of IP-addresses that can be used for direct robot-to-robot communication. Each team will also be allocated a limited range of UDP ports on which broadcast will be permitted.

The GameController will use UDP to connect to the robots. The source distribution of the GameController provides the header file \emph{RoboCupGameControlData.h} that defines all messages sent by the GameController to the robots. They correspond to the \emph{robot states} described in Section~\ref{sec:robot_states}.

The use of remote processing/sensing is prohibited.

\section{Game Process}

\subsection{Structure of the Game}
\label{sec:game_struct}

A game consists of three parts, \ie the first half, a half-time break, and the second half. Each half is 10 minutes. The clock stops during stoppages of play\footnote{This may not be the case during the preliminary games.} (such as kick-offs after goals). The extra time over ten minutes total is referred to as ``lost time''. The half-time break is also ten minutes, during this time both teams may change robots, change programs, or anything else that can be done within the time allotted. In the preliminaries a game can finish in a draw as no penalty shoot-out will follow. In the finals or in the intermediate round a game that ends in a draw will be followed by a penalty shoot-out (see Section~\ref{sec:penalty_shoot-out}).

The teams will change the goal defended and color of the team markers during the half-time break.

\begin{figure}[t]
\centerline{\includegraphics[width=0.9\columnwidth]{figs/states.pdf}}
\caption{Robot states. Button interface transitions are shown in gray. GameController transitions are shown in black. However, any transition possible can actually be sent by the GameController.}
\label{fig:robot_states}
\end{figure}

\subsection{Robot States}
\label{sec:robot_states}

Robots can be in six different states (\cf Figure~\ref{fig:robot_states}). If the wireless is available, these states will be set by the GameController. Teams must implement code to receive and correctly respond to wireless GameController packets, and also give indication of the game state, team color, and the kickoff state.
%RHM 4/7/07
\emph{If a robot does not respond to either the game controller or the button press interface, then it is not included in the game, and the game starts without the offending robots.}

\begin{description}

\item[Initial.] After booting, the robots are in their \emph{initial} state. In this state, the button interface for manually setting the team color and whether the team has kick-off is active. The robots are not allowed moving in any fashion besides initially standing up. Pressing the left foot bump sensor will switch the team color. Shortly pressing the chest button will switch the robot to the \emph{penalized} state.

\item[Ready.] In this state, the robots walk to their legal kick-off positions (\cf Section~\ref{sec:kick-off}). They remain in this state, until the head referee decides that there is no significant progress anymore (after a maximum of \KickOffAutoTime). This state is not available if only the button interface is implemented.

Robots may be disentangled by the referees at the start of the Ready state. After that, any robots which are close to each other (\cf Section~\ref{sec:player_pushing}) will be placed manually to the positions shown in Figure~\ref{fig:ko}.

\item[Set.] In this state, the robots stop and wait for kick-off (\cf Section~\ref{sec:kick-off}). If they are not at legal positions, they will be placed manually by the assistant referees to the positions shown in Figure~\ref{fig:ko}. They are allowed to move their heads before the game (re)starts but are not allowed moving their legs or locomote in any fashion. This state is not available if only the button interface is implemented. Robots that do not listen to the GameController will be placed manually. Until the game is (re)started, they are in the \emph{penalized} state.

\item[Playing.] In the \emph{playing} state, the robots are playing soccer. Shortly pressing the chest button will switch the robot to the \emph{penalized} state.

\item[Penalized.] A robot is in this state when it has been penalized. It is not allowed moving in any fashion, \ie also the head has to stop turning. Shortly pressing the chest button will switch the robot back to the \emph{playing} state.

\item[Finished.] This state is reached when a half is finished. This state is not available if only the button interface is implemented.

\end{description}

The team color should be displayed during the whole game on the LED of the left foot (blue/red). Teams that support the GameController can visualize whether the robot's team has kick-off on the LED of the right foot (off/white) in the states \emph{initial}, \emph{ready} and \emph{set}. The current game state should be displayed on the LED in the torso. The colors corresponding to the game states are:

\begin{itemize}

\item Initial: Off

\item Ready: Blue

\item Set: Yellow

\item Playing: Green

\item Penalized: Red

\item Finished: Off

\end{itemize}

\subsection{Goal}

A goal is achieved when the entire ball (not only the center of the ball) goes over the goal-side edge of the goal line, \ie the ball is completely inside the goal area\footnote{The goal line is part of the field.}. The restart after the goal shall adopt the same rules as the kick-off.

Note that a goal can never be awarded where the last contact of the ball with a robot was by the arm or hand (even if unintentional -- see also Section~\ref{sec:hand_ball}) of an attacking robot. Should the ball enter the goal area where the last contact is accidental contact with the arm or hand of an attacking robot, the goal shall not count and a goal kick is awarded, that is, it shall count as if the ball is out by the attacking team (see Section \ref{sec:throw_in}).

\subsection{Applying Penalties}

See Section~\ref{sec:penalty_procedure}.

\subsection{Kick-off}
\label{sec:kick-off}

For kick-off, the robots listening to the wireless GameController run through three states: \emph{ready}, \emph{set}, and \emph{playing}. Robots not listening to the GameController are simply penalized and manually placed for kick-off\footnote{Note that robots being manually placed because they are not listening to the GameController must still be placed in the restricted set of legal positions for manually placed robots. It is to a team's advantage to have their robots listen to the GameController.}.

In the ready state, the robots should walk to their legal kick-off positions. These positions are always located inside their own side of the field. No player is allowed touching the halfway line.
One field player of the attacking team can walk to a position inside the center circle.
The other field players of the attacking team can walk to any position within their own half, except for the penalty area.
The field players of the defending team have to be located in the half of their field that is behind the penalty mark (none of their feet are allowed to be past the penalty mark), and their feet must be outside the penalty area.
In contrast, the feet of both goal keepers must be \emph{inside} the penalty area.

If robots collide during the autonomous placement, the ``Player Pushing'' rules are applied, but the penalty is manual placement by the assistant referees.

The robots have a maximum of \KickOffAutoTime to reach their positions. If all the robots have reached legal positions and have stopped, or if \KickOffAutoTime have passed, the robots will be switched into the \emph{set} state, in which they must stop walking. Each robot that is not at a legal position at this point in time will be placed manually by the assistant referees to the positions as shown in Figure~\ref{fig:ko}. Robots that are legally positioned will not be moved by the assistant referees unless a manual position is requested by the team leader.
%RHM: 4/7/09
\emph{In the case where the team leader requests manual placement, all robots on that team are manually positioned.}


\begin{figure}[t]
\centerline{\includegraphics[width=\columnwidth]{figs/manual_placement_2011.png}}
\caption{Manual setup for kick-off.}
\label{fig:ko}
\end{figure}

There are extra restrictions on the legal positions of manually positioned robots. The kicking-off robot is placed on the center circle, right in front of the penalty mark. Its feet touch the line, but they are not inside the center circle. 
One of the other field players of the attacking team is placed in front of one of the corners of the penalty box on the height of the penalty mark (i.e. the virtual intersection point of the corner of the penalty box and the penalty mark), and the other one is placed on the other corner of the penalty box without having its feet touching the corner. 
The two field players of the defending team are placed on the halfway between the corners of their own penalty box and the sidelines. The third field player is placed in front of the penalty box, halfway between the center of the penalty box and one of the goal bars.
As autonomously placed robots are allowed to be much closer to the ball, successful autonomous placement results in a significant advantage over manual placement.

Just before the \emph{set} state is called, the ball is placed on the center point of the center circle by one of the referees. If it is moved by one of the robots before set is called it is replaced by one of the referees.

After the head referee has signaled the kick-off, the robot's state is switched to \emph{playing} (again either by the GameController or manually), in which they can actually play soccer.

Note that a goal can never be scored directly from a shot from the kick-off. See Section~\ref{sec:kick-off_shot} for details.

If the assistant referees have misconfigured the robots (\eg they set the wrong team color), the kick-off is repeated. In this case, goals scored with at least one misconfigured robot on the field are not counted. The time that was played with a wrong configuration is counted as ``lost time'', \ie the half should be lengthened by it. Note that the assistant referees are only responsible for setting team color on kick-off. Robots replaced after a request for pickup are the responsibility of the team, and should be handed to the referees or assistants in the `penalized' state.

The current GameController requires robots to know both their team number and their robot number within the team. It is each team's responsibility to make sure this is correctly configured. It is recommended that the robot indicates its number within the team on bootup so that this can be easily checked at the start of the game.

\subsection{Free Kick}

None.

\subsection{Penalty Kick}
\label{sec:penalty_kick}

A penalty kick is carried out with one attacking robot and one opposing goal keeper. Other robots should be powered off and stay outside of the field. Teams are allowed to switch to specially designed software for a penalty kick. Standard penalty kicks are taken against the opponent goal. For a penalty shootout, see Section~\ref{sec:penalty_shoot-out}.

The ball is placed on the penalty spot, at the end of the field closest to the goal being defended. The attacking robot is positioned at the center of the field, facing the ball. The goal keeper is placed with its feet on the goal line and in the center of the goal.

Neither robot shall move their legs before the penalty kick starts. Movements of the robot's head and arms are allowed as long as the robot does not locomote. Technically, the robots are in the \emph{set} state when waiting for the penalty kick to start. If robots are not listening to the GameController, they are in the \emph{penalized} state instead. The robots are started by switching to the \emph{playing} state.

The penalty kick ends when the kicker scores the goal, the time expires, or the ball leaves the field. The time limit for the kicker is 1 minute after the penalty kick starts. The ball must be in the goal within this time limit in order to count as a goal.

The goal keeper is not allowed to touch ball that is completely outside the penalty area and the attacking robot is not allowed to touch a ball that is completely inside the penalty area. The line is part of the penalty area. If the attacker touches the ball when the ball is inside the penalty area then the penalty shot is deemed unsuccessful. If the goal keeper touches the ball when the ball is outside the penalty area then a goal will be awarded to the attacking team.

All the rules such as ``Ball Holding'', ``Pushing'' and others are also applied during the penalty kick. The only exception is the ``Illegal Defender'' rule, \ie the penalty shooter is allowed to enter its own penalty area.
A goal keeper will not be penalized for inactivity during a penalty kick (including penalty shoot out), provided its stiffness is on. Other penalties are applied as usual.

\subsection{Penalty Kick Shoot-out}
\label{sec:penalty_shoot-out}

A penalty kick shoot-out is used to determine the outcome of a tied game when an outcome is required (for example, during quarter or semi finals). 
There will be a five minute break between the end of the game and the start of the penalty kicks.
In the preliminary rounds, the penalty kick shoot-out will consist of three penalty shots per team; in the quarterfinals and later, it will consist of five penalty shots per team.
All penalty shots are \emph{taken against the blue goal}. At the conclusion of these shots the team that has scored the most goals will be declared the winner. Note that a winner can be declared before the conclusion of the penalty shoot-out if a team can no longer win, for example, a team requires 3 goals to win but only has 2 attempts remaining. If the two teams still remain tied then a sudden death shoot-out will follow until a definite winner is found.

The procedure for each attempt is the same as for the normal penalty kick described in Section~\ref{sec:penalty_kick}. For the first five attempts, the standard time limit of 1 minute is applied. If after five penalty kicks by each team there is no result (that is, each team has scored the same number of goals), then the decision will be made by the following sudden death shoot-out procedure.

Penalty shoot-outs are taken against the \emph{blue} goal. The GameController will configure the team color of the penalty shooter as red and the team color of the goal keeper as blue.

\subsubsection{Sudden Death Shoot-Out}

The time limit for sudden death penalty shots is two minutes.

These attempts will be timed (that is, for a goal scored, how long did it take to score the goal) and measured (that is, if a goal is not scored, what is the shortest distance between the ball and the goal line segment between the goal posts, achieved at any time during the penalty shot) by the referee. After these attempts, the game decision will be made as follows:

\begin{enumerate}

\item If only one team scores a goal, that team wins.

\item If both teams score a goal, then if one team is timed to have scored at least 2 seconds faster than the other team, the faster team wins. Otherwise, the sudden death shoot-out is repeated.

\item If neither team scores a goal, then if one team is measured to have moved the ball more than 50~mm closer to the goal than the other team, the closer team wins. Otherwise, the sudden death shoot-out is repeated.

\item If neither team has touched the ball during the shoot-out, the referee will toss a coin to decide the game.

\end{enumerate}

\subsection{Throw-in}
\label{sec:throw_in}

A ball is considered to have left the field when there is no part of the ball over the outside of the boundary line (\ie the line itself is in). If the ball leaves the field it will be replaced on the field by an assistant referee. There is \emph{no} stoppage in play.

If the ball goes over a side-line then the assistant referee will replace the ball back on the field on the throw-in line on the same side of the field as the ball went out of play.

The ball will be replaced on the throw-in line at the farthest back of these two locations: a) one meter back from the point it went out or b) one meter back from the location of the kicking robot. We define `back' as being towards the goal of the team that last touched the ball. Note that if the one meter placement would cause the ball to be placed off the end of the throw-in line, then it should be placed at the end of the throw in line, and not beyond.

If the ball goes over an end-line then the assistant referee will replace the ball back on the field according to the following rules:

\begin{itemize}

\item If the ball was last touched by the defensive team then the ball is replaced on the closest endpoint of the throw-in line.

\item If the ball was touched by the offensive team, the ball is replaced on
the throw-in line at the farthest back of these two locations: a) one
meter back from the location of the kicking robot, or b) at the halfway
line.

%If the ball was touched by the offensive team, the the ball is replaced at on the throw-in line at the farthest back of these three locations: a) one meter back from the point it went out, b) one meter back from the location of the kicking robot, or c) at the halfway line.

\end{itemize}

Balls are deemed to be out based on the team that last touched the ball, irrespective of who actually kicked the ball.

\paragraph{Example 1.} The red goalie kicks the ball out the end of the field to the right of the yellow goal. The ball is placed on the endpoint of the throw-in line to the right of the yellow goal.

\paragraph{Example 2.} A blue robot on the yellow side of the field kicks the ball out the end of the field to the right of the yellow goal. The ball is placed on the intersection of the right throw-in line and the halfway line.

\paragraph{Example 3.} A blue robot on the blue side of the field kicks the ball out the end of the field to the right of the yellow goal. The ball is placed on one meter behind the robot on the right throw-in line.

\paragraph{Example 4.} A blue robot at midfield kicks the ball over the left sideline 2 meters into the yellow half of the field. The ball is replaced on the left throw-in line 1 meter into the blue half of the field (one meter behind the robot).

\paragraph{Example 5.} A blue robot at midfield kicks the ball over the left sideline 2 meters into the blue half of the field (towards its own goal). The ball is replaced on the left throw-in line 3 meters into the blue half of the field.

\paragraph{Example 6.} A blue robot kicks the ball but the ball touches a red robot at midfield before leaving the field near the centre line. The ball is regarded as out by red and therefore is replaced on the throw-in line 1 meter closer to the yellow goal.


\subsection{Game Stuck}
\label{sec:game_stuck}

In the event of no substantial change in the game state for 15 seconds, this is considered a game stuck. 
This includes if a robot is at the ball circling it for 15 seconds without kicking the ball.
The main referee has two options how to solve the game stuck and to reestablish the chance of progress in the game. The intention of the game stuck rule is to achieve progress with as little intervention as possible, \ie the \emph{Local Game Stuck} rule will be preferred, but only if there is a chance that its application will result in progress in the game.

\subsubsection{Local Game Stuck}

The referee picks up the nearest robot to the ball and moves the robot to the half way line. The referee does \emph{not} replace the ball. If the ball is accidentally bumped when removing the robots, the ball should be replaced where it was when the game stuck was called. As a special exception, if the goalie is involved in a game stuck situation while having both feet in its own penalty area, it will be placed on the penalty mark of its half facing its goal.

\subsubsection{Global Game Stuck}

The referee stops the game and restarts the game from the kick-off formation. The kick-off will be awarded to the team defending the side of the field the ball is on when the game stuck is called. A global game stuck can only be called if at least one robot has touched the ball since the previous kick-off.

\subsection{Request for Pick-up}

Either team may request that one of their players be picked up only for hardware dysfunction and software crashes at any point in the game (called ``Request for Pick-up''). It is permitted to change batteries, fix mechanical problems, or, if necessary, reboot the robots, but not to change or adjust their program. Any strategic ``Request for Pick-up'' is not allowed. The head referee will indicate when the robot is no longer affecting play and can be removed from the field by an assistant referee. The robot will be replaced on the half way line after a minimum of 30 seconds after it was taken off the field following the normal replacement procedure used after the standard removal penalty (see Section~\ref{sec:removal_penalty}).

If a robot has been rebooted and the wireless is not working, it is the responsibility of the team members (not the assistant referees) to configure its team color correctly. The robot should be returned to the assistant referees in the \emph{penalized} state so that the assistant referees cannot accidentally change the robot's team color.

\subsection{Request for Timeout}

At any stoppage of play (after a goal, stuck game, before half, etc.) either team may call a timeout. Each team can call a maximum of 1 timeout per game with a total time totaling no more than 5 minutes. During this time, both teams may change robots, change programs, or anything else that can be done within the time allotted. The timeout ends when the team that called the timeout says they are finished, at which time they must be ready to play. At this time the other team must either be ready to play or call a timeout of its own. The clock stops during timeouts, even during the preliminaries.

After the completion of the timeout, the game resumes with a kick off for the team which did not call the timeout.

If a team is not ready to play at the assigned time for a game, the referee will call the timeout for that team. After the expiration of such a timeout, if the team is still not ready to play then they must either forfeit the game, or the referee shall start the game with only one team on the field.  The team that wasn't ready can return its robots to the field as per the rules for ``Request for Pick-up''. If both teams are not ready, the referee will call timeouts for both teams. This ``double timeout'' expires after 10 minutes.

\subsection{Winner and Rankings}

The team which scored more goals than the other is the winner of the match. If the two teams scored the same number of goals, the game will be a draw. The draw will follow the same system defined in Section~\ref{sec:game_struct}. Total (and final) standings will be decided on points as follows (the points will be given based on the result of each game):

\makebox[\columnwidth]{ \hfill Win = 3 pts\hfill Draw = 1 \hfill
Lose = 0 pts\hfill }

If a team's obtained points is the same as another team's after the preliminary round is complete, the following evaluations will be applied in order to qualify the finalists.

\begin{enumerate}

\item The points obtained

\item The
%RHM %average
 difference between goals for and goals against per game

\item The average goals for per game

\item Game result between the teams directly

\end{enumerate}

\subsection{Rules for Forfeiting}
\label{sec:forfeit}

If a team chooses to forfeit a match then the opposing team will play the match against an empty field. The result will stand, even if the team fails to score against an empty field or scores an own goal. Teams may choose to forfeit games at any stage. Any game with a final score differential greater than 10 points will be considered a forfeit.

\section{Forbidden Actions and Penalties}
\label{sec:forbidden_act}

The following actions are forbidden. In general, when a penalty applies, the robot shall be replaced, not the ball. For penalties that are timed, the penalty time is considered to be over whenever the game time stops (for goals, half-time, and game stuck).

\subsection{Locomotion Type}
\label{sec:locomotion_type}

Robots should clearly demonstrate bipedal walking similar to human walk. Other types of locomotion involving other parts than feet (crawling etc.) are strictly forbidden. It is duty of the head referee to decide whether a robot's locomotion is appropriate.

\subsection{Penalty procedure}
\label{sec:penalty_procedure}

When a robot commits a foul, the head referee shall call out the infraction committed, the jersey color of the robot, and the jersey number of the robot. Each robot will be labeled with a jersey number before the game. The penalty for the infraction will be applied immediately by an assistant referee. The assistant referees should perform the actual movement of the robots for the penalty so that the head referee can continue focusing on the game. The operator of the GameController will send the appropriate signal to the robots indicating the infraction committed.

\subsection{Standard Removal Penalty}
\label{sec:removal_penalty}

Unless otherwise stated, all infractions in this league result in the removal of the infringing robot from the field of play for 30 seconds, after which it will be returned to the field of play. This process is called the standard removal penalty, and a detailed description of the process follows.

When the head referee indicates a foul has been committed that results in the standard removal penalty, the assistant referee closest to the robot will remove the robot immediately from the field of play. The robot should be removed in such a way as to minimize the movement of the other robots and the ball. If the ball is inadvertently moved when removing the robot, the ball should be replaced to the position it was in when the robot was removed.

The operator of the GameController will send the appropriate signal to the robot indicating the infraction committed. If the wireless is not working and the penalty is timed, the assistant referee handling the robot will reset the robot into the \emph{penalized} state for the duration of the penalty. This may not be done if the penalty is not timed, \ie it is a 0 seconds penalty. After a penalty is signaled to the robot, it is not allowed to move in any fashion, such as being in the \emph{initial} state. The removed robot will be placed outside of the field facing away from the field of play.

The GameController will keep track of the time of the penalty. The operator of the GameController will signal the assistant referees when the penalty is over, so that one of them can put the robot back on the field. The assistant referee will then place the robot on the field on the halfway line as close to the sideline as possible. The robot should be pointed towards the opposite sideline. The robot should be placed on the side of the field furthest from the ball. If there is another robot already in this position, the robot should be replaced at a nearby location along the sideline facing towards the opposite sideline. If there are no practical locations nearby, a location along one of the sidelines should be found that is away from the ball (the robot should be set down facing the opposite sideline). When finding a nearby location, locations away from the ball should be preferred.

When the robot is on the field again, the operator of the GameController will send the \emph{playing} signal to it. If the wireless is not working, the assistant referee who placed the robot back on the field has to bring it into the \emph{playing} state again.

\subsection{Manual Interaction by Team Members}

Manual interaction with the robots, either directly or via some communications mechanism, is not permitted. Team members can only touch one of their robots when an assistant referee hands it over to them after a ``Request for Pick-up''.

\subsection{Kick-off Shot}
\label{sec:kick-off_shot}

A ``kick-off shot'' can never score a goal. A ``kick-off shot'' is a shot taken by the team kicking off after a kick-off before the entire ball has left the center circle, including the boundary line. If a kick-off shot enters the goal (either directly or via contact with an opposing robot), no goal will be scored and a kick-off will be awarded to the defending team (as per Section~\ref{sec:kick-off}).

\subsection{Ball Holding}
\label{sec:ball_holding}

The goalie is allowed to hold the ball for up to 5 seconds as long as it has one foot inside in its own penalty area. In all other cases, robots are allowed to hold the ball for up to 3 seconds. Holding the ball for longer than this is ``Ball Holding'' and is not allowed.

A robot which does not leave enough open space around the ball will be penalized as ``Ball Holding'' if that situation continues more than 3 seconds. The occupation of the ball is judged using the convex hull of the projection of the robot's body onto the ground. ``Enough open space'' means that at least the half of the ball is not covered by the convex hull. It is not important whether the robot actually touches the ball.


\begin{figure}[t]
\centerline{\begin{tabular}{ll}
a) & b) \\
\includegraphics[scale=0.7]{figs/holding1} &
\includegraphics[scale=0.7]{figs/holding4}
\end{tabular}}

\caption{Examples for ``Ball Holding''. The orange circle is the ball, the blue polygon visualizes the convex hull of the robot's projection onto the ground and the red area shows the occupied portion of the ball. Situations a) is legal, whereas b) violates the rule.}
\label{fig:holding}
\end{figure}

Intentional continual holding is prohibited even if each individual holding time does not continue for up to the time limit. In general robots should release the ball for approximately as long as they were holding it to reset the clock. Without a sufficient release, the continual holding is regarded as a continuous hold from the very beginning and the holding rule is strictly applied. The violation of this rule will result in the standard removal penalty (see Section~\ref{sec:removal_penalty} for details). The ball should be removed from the possession of the robot and placed where the foul occurred. If the robot that held the ball has moved the ball before the robot can be removed, the ball shall be replaced where the foul occurred.

\paragraph{Example.} A robot holds the ball and before the referees
can remove the robot, it shoots the ball into the goal. The goal
will not be counted and the ball will be replaced where the robot
held the ball.

\subsection{Fallen or Inactive Robots}
\label{sec:fallenrobots}

If a robot falls during the game, it should start executing a getup action within 5 seconds. If it does not commence a get up action within 5 seconds, it will be removed as per the standard removal penalty. A robot which is unable to autonomously stand up within 20 seconds after a fall will be removed and subject to the standard penalty. The goal keeper, inside its own penalty area, is the only robot permitted to `dive' (that is deliberately fall in a way that might cause its torso, arms or hands) to intercept the ball. In all other cases, the robot should be programmed to attempt to remain upright -- that is, supported by its feet.

A robot that has ceased activity for 10 seconds or has turned off will be removed by the referees and is subject to the standard removal penalty. A robot is active if it performs at least one of the following:

\begin{enumerate}

\item The robot walks in any direction, or turns.

\item The robot searches for the ball, or is looking at the ball.

\end{enumerate}

\paragraph{Note:} The intention of this rule is not to penalize robots simply for being stationary -- provided they are not `asleep' and have not `crashed'.

\subsection{Player Stance}
\label{sec:player_stance}

Robots are not allowed to stay in a stance that is wider than the width of the robot's shoulders for more than 5 seconds. The robot is allowed to go into a wide stance as long as it comes back to a normal stance within 5 seconds. Staying in a wide stance for longer than 5 seconds will result in the standard penalty. If the robot has fallen down, it must start getting up within 5 seconds. 

%This rule is to prevent teams from using stationary keepers that stay in a wide blocking stance. 

\newpage
\subsection{Player Pushing}
\label{sec:player_pushing}

Except for the situations described below in Section \ref{sec:not_pushing}, contact between two robots of the same team is considered pushing, and between robots of opposing teams, each contact of a robot with another robot that is forceful enough to make the other robot fall is considered pushing. In addition, a robot that walks front-to-back into another is considered pushing, as is walking into another robot for 3 seconds, regardless of the `pushing force'. Two robots may be penalized for pushing from one incident, for example, if they collide front to front whilst both are moving.

\subsubsection{Contact between teammates}

Contact between 2 robots of the same team is considered pushing. 2 robots of the same team are considered to be in contact if they are both standing (i.e., not fallen), and:
\begin{enumerate}
\item Any of their parts are touching, i.e., in direct contact, or
\item They are connected through contact with a robot of the opposing team (e.g., if $A_1$ is touching $B_1$ and $B_1$ is touching $A_2$, then $A_1$ and $A_2$ are considered to be in contact).
\end{enumerate}

The robot that gets penalized is chosen in the following order:
\begin{enumerate}
\item If one robot is a goal keeper in its penalty area, the other robot is penalized;
\item If one robot is stationary, the other one is penalized;
\item Otherwise, the robot closer to the ball is penalized.
\end{enumerate}

The situations described in Section \ref{sec:not_pushing} are exceptions to the descriptions above. For example, neither robot is penalized in the following scenarios:
\begin{enumerate}
\item Both robots are stationary.
\item One robot is a goal keeper in its penalty area, and the other is stationary.
\item One robot is attempting to kick the ball, and the other is stationary.
\end{enumerate}

\subsubsection{Contact between opposing robots}

Robots of opposing teams are allowed contact in certain scenarios:

\begin{enumerate}

\item Shoulder-to-shoulder or arm-to-arm contacts are allowed when both robots are proceeding to the ball.

\item Chest-to-chest contact is allowed when the ball is between the robots. If the ball is not between the robots then both robots are penalized.

\end{enumerate}

The following forms of contact are considered pushing, except the scenarios listed above, and in Section~\ref{sec:not_pushing}:

\begin{enumerate}

\item Any form of contact that has enough force to knock a robot off-balance, even chest-to-shoulder/arm or shoulder/arm-to-chest.

\item Walking into another robot for 3 seconds, even if the `force to push' is minimal.

\item Walking front-to-back into another robot, even if the `force to push' is minimal.
\end{enumerate}

\subsubsection{Pushing concerning a fallen robot}

A robot who has fallen over, should be allowed to attempt a `getup'
routine. While doing the getup movement the standard pushing rules will be applied. Examples:

\begin{enumerate}

\item If the `getting up robot' makes contact with an opposing robot who is not moving towards the fallen robot then the fallen robot will be called for pushing. (Basically saying if the opposing robot has `cleared' the area of the fallen robot, however the getup routine causes the contact then the getting up robot should be penalized.)

\item If during the getup an opposing robot moves towards the fallen robot and makes contact with the fallen robot who is attempting a getup, then the standing robot should be penalized.

\end{enumerate}

\subsubsection{Situations where pushing does not apply}
\label{sec:not_pushing}

\begin{enumerate}

\item A stationary robot cannot be penalized for pushing.

\item A robot will not be penalized for pushing whilst attempting to kick the ball, provided that the ball is close enough to the robot that in the opinion of the referee the attempted kick could contact the ball.

\item The goal keeper, whilst in its own penalty area, and either moving to the ball or looking at the ball, will not be penalized for pushing.

%\item A robot cannot be penalized for pushing if the contact is with a robot that is ruled to have deliberately moved to obstruct.

%\item Two robots from different teams facing each other with the ball in between them cannot be penalized for pushing \textit{each other}.

%\item Two robots from different teams moving towards the ball that make shoulder-to-shoulder or arm-to-arm contact cannot be penalized for pushing \textit{each other}.

\end{enumerate}

\subsubsection{Results of Player Pushing}
\label{sec:pushing_results}

The standard removal penalty will apply for pushing. \emph{The 4th time a team is called for pushing in a half, the offending robot will be removed for the remainder of the half. Subsequently for the half, a robot from the team will be removed for every 2 pushing offenses (i.e., the 6th, and 8th times).} If the ball is moved as the result of pushing, then it will be replaced to where it was at the time of the infringement.

If a robot illegally pushes an opposing robot that is not pushing, and the collision causes the opposing robot to fall, the fallen robot will be picked up by one of the referees and replaced in an upright position at the point at which it fell. This is only required if the robot cannot stand up on its own.


%------------------------------------------------------------
% tried to define player pushing in a less ambiguous and more compact way
\subsection{Player Pushing (2)}
\label{sec:player_pushing2}
\colorbox{yellow}{(Overview instead of giving a short definition to be extended/corrected/contradicted later)}

Section~\ref{sec:situations_no_pushing} defines situations where pushing does not apply.
Those exceptions overrule all other conditions.
In all other situations, a \textit{pushing contact} between two robots is considered player pushing and penalized according to section~\ref{sec:pushing_results}.
Sections~\ref{sec:pushing_own} to~\ref{sec:pushing_fallen} define such pushing contacts between robots in the same team, robots of opposing teams, and with fallen robots, respectively.

\subsubsection{Situations Where Pushing Does Not Apply}
\label{sec:situations_no_pushing}
\colorbox{yellow}{(Unchanged)}

\begin{enumerate}
	\item A stationary robot cannot be penalized for pushing.
	\item A robot will not be penalized for pushing whilst attempting to kick the ball, provided that the ball is close enough to the robot that in the opinion of the referee the attempted kick could contact the ball.
	\item The goal keeper, whilst in its own penalty area, and either moving to the ball or looking at the ball, will not be penalized for pushing.
\end{enumerate}

\subsection{Contact between Teammates}
\label{sec:pushing_own}
\colorbox{yellow}{(Removed repeating definitions and simplified the formulation)}

Any contact between 2 robots of the same team is a potential pushing contact if they are both standing (i.e., not fallen), and:
\begin{enumerate}
\item Any of their parts are touching, i.e., in direct contact, or
\item They are connected through contact with a robot of the opposing team (e.g., if $A_1$ is touching $B_1$ and $B_1$ is touching $A_2$, then $A_1$ and $A_2$ are considered to be in contact).
\end{enumerate}

Out of those teammates which are involved and do not fall under any of the exceptions in section~\ref{sec:situations_no_pushing}, the one closer to the ball is penalized.
If the exceptions of section~\ref{sec:situations_no_pushing} apply to all involved teammates, then no robot gets penalized.

\subsection{Contact between Opposing Robots}
\label{sec:pushing_opponent}
\colorbox{yellow}{(Mostly unchanged)}

Robots of opposing teams are allowed contact in certain scenarios:
\begin{enumerate}
	\item Shoulder-to-shoulder or arm-to-arm contacts are allowed when both robots are proceeding to the ball.
	\item Chest-to-chest contact is allowed when the ball is between the robots. If the ball is not between the robots then both robots are penalized.
\end{enumerate}

The following forms of contact are considered pushing contacts, except the scenarios listed above and in section~\ref{sec:situations_no_pushing}:
\begin{enumerate}
	\item Any form of contact that has enough force to knock a robot off-balance, even chest-to-shoulder/arm or shoulder/arm-to-chest.
	\item Walking into another robot for 3 seconds, even if the `force to push' is minimal.
	\item Walking front-to-back into another robot, even if the `force to push' is minimal.
\end{enumerate}



\subsection{Contact with a Fallen Robot}
\label{sec:pushing_fallen}
\colorbox{yellow}{(Changed to not only cover opposing robots)}

A robot who has fallen over, is allowed to attempt a `getup' routine, but must not interfer with other stationary robots.
While doing the getup movement the following pushing contacts might occur:

\begin{enumerate}
	\item If the `getting up robot' makes contact with another robot which is not moving towards the fallen robot, then the fallen robot will be called for pushing. (Basically saying if the opposing robot has `cleared' the area of the fallen robot, however the getup routine causes the contact, then the getting up robot should be penalized.)
	\item If during the getup another robot moves towards the fallen robot and makes contact with the fallen robot who is attempting a getup, then the upright robot will be penalized if non of the exceptions of section~\ref{sec:situations_no_pushing} applies.
\end{enumerate}


\subsection{Results of Player Pushing}
\colorbox{yellow}{(Unchanged)}

The standard removal penalty will apply for pushing. \emph{The 4th time a team is called for pushing in a half, the offending robot will be removed for the remainder of the half. Subsequently for the half, a robot from the team will be removed for every 2 pushing offenses (i.e., the 6th, and 8th times).} If the ball is moved as the result of pushing, then it will be replaced to where it was at the time of the infringement.

If a robot illegally pushes an opposing robot that is not pushing, and the collision causes the opposing robot to fall, the fallen robot will be picked up by one of the referees and replaced in an upright position at the point at which it fell. This is only required if the robot cannot stand up on its own.

\subsection{Playing with Arms/Hands}
\label{sec:hand_ball}

A field player or a goal keeper outside its own penalty box that intentionally touches the ball with its arms/hands in a manipulative manner (i.e. to stop the ball, to kick the ball etc.) will be subject to the standard removal penalty. In this case, the ball is to be replaced at the point where it contacted the arms/hands of the offending robot.

\subsection{Damage to the Field}

A robot that damages the field will be removed from the field for the remainder of the game. Similarly a robot that poses a threat to the spectator's safety will also be removed. In such a case, a normal penalty kick will be awarded to the opposing team (\cf Section~\ref{sec:penalty_kick}).

\subsection{Leaving the Field}
\label{sec:leaving_field}

A robot that intends to leave the \TotalWidth $\times$ \TotalLength carpeted area will be subject to the standard removal penalty (see
Section~\ref{sec:removal_penalty}). This penalty can already be called after a robot leaves the 4~m $\times$ 6~m playing field if the robot appears to be ``lost''.

In addition, a robot that walks into the goal net for more than 5 seconds will also be subject to the standard removal penalty.

\subsection{Illegal Defender}

Only the goalie can be within its team's penalty area. Having both legs inside the penalty area (\ie at least touching the line) is the definition of being in the penalty area and that situation is not allowed for defending field players. When other defending robots enter the area, they will be subject to the standard removal penalty (see Section~\ref{sec:removal_penalty}). This is called the ``Illegal Defender Rule''. This rule will be applied even if the goalie is outside of the penalty area, but not if an operational defender is pushed into the penalty area by an opponent.

If an illegal defender kicks an own goal, the goal is scored for the opponent. If there is any doubt about whether a goal should count (e.g. the illegal defender infraction is called, but the robot scores the own goal immediately afterwards, before it is removed) then the decision shall be against the infringing robot.

%\subsection{Obstruction}
%
%When a robot that is not heading for the ball is actively and intentionally blocking the way to the ball for a robot of the other team, this is an obstruction (\cf Figure~\ref{fig:obstruction}). The obstructing robot will be subject to the standard removal penalty (see Section~\ref{sec:removal_penalty}). A robot will not be penalized for pushing a robot that is penalized for obstruction (see Section~\ref{sec:player_pushing}).
%
%\paragraph{Example.} If a robot of one team, robot A, is heading for the ball, but a robot of the other team, robot B, is in its way without heading for the ball, this, by itself, is not an obstruction. But, if robot A starts to move around robot B, and then robot B intentionally moves to block robot A again, this is an obstruction, even if the robots do not actually touch.
%
%Note: The intent of the obstruction rule is to stop people implementing code that deliberately attempts to obstruct opponent robots.
%
%\begin{figure}[t]
%\centerline{\includegraphics[height=5cm]{figs/obstruction}}
%\caption{Obstruction}\label{fig:obstruction}
%\end{figure}

\subsection{Jamming}

During the match any robot shall never jam the communication and the sensor systems of the opponents:

\begin{description}

\item[Wireless communication.] Teams can use a bandwidth of up to 500 Kbps of the wireless. This includes any data transferred, namely the actual payload and any protocol overhead created, \eg, by TCP or UDP. If a team uses more bandwidth over a couple of seconds in a game, it will be disqualified for that game. Except for the wireless cards and the access points provided by the organizers of the competition, nobody close to the field is allowed using 2.4~GHz radio equipment (including cellular phones and/or Bluetooth devices).

\item[Acoustic communication.] If acoustic communication is used by both teams, they shall negotiate before the match how they can reduce interference. If only one team uses acoustic communication, the robots of the other team shall avoid producing any sound. In addition, both the teams and the audience shall avoid intentionally confusing the robots by producing similar sounds to those used for communication.

\item[Visual perception.] To avoid confusing other robots, the robots are not allowed to switch LEDs to orange. In general, the use of flashlights is not allowed during the games.

\end{description}

\section{Judgement}

The referees are the only persons that are allowed inside the playing area.

\subsection{Head Referee}

The head referee signals game starts, restarts, when a goal was scored, the case of \emph{game stuck}, and penalties by a single whistle. In general, the head referee first whistles and then announces the reason for the whistle. The only exception is the case of the kick-off, in which the reason for the whistle is obvious. The whistle defines the point in time at which the decision is made. If the head referee has to announce many decisions in short sequence, he may skip whistling. For penalties, he announces the infraction committed, the team color, and the jersey number of the robot, \eg ``illegal defender, blue number 3''. In case of a goal scored, local or global game stuck, this is also announced verbally. By two whistles, the head referee terminates the first half; by three whistles he terminates the second half, \ie the whole game. The head referee is also responsible for keeping the time of each half, \ie, he or she stops the clock after a goal or game stuck, and continues it at the kick-off\footnote{The clock may not be stopped during the preliminaries.}. The head referee may choose to delegate this task to the GameController, it should be noted that the ultimate responsibility for correct time keeping still remains with the head referee.

In the penalty kick shoot-out, the head referee keeps the time.

Any decision of the head referee is valid. There is no discussion about decisions during the game, neither between the assistant referees and the head referee, nor between the audience or the teams and the head referee.

\subsection{Assistant Referees}

The two assistant referees handle the robots. They start them if the wireless is not working, they move them manually to legal kick-off positions, they take them out when the robots are penalized, and they put them in again. If a team requests picking up a robot, an assistant referee will pick it up and give it to one of the team members. An assistant referee will also put the robot back on the field. In addition, the assistant referees can indicate violations against the rules committed by robots to the head referee, so that the head referee can decide whether to penalize a certain robot or not. Assistant referees should only enter the field to execute a decision made by the main referee. They should not prevent robots from falling during the game.

\subsection{Operator of the GameController}

The operator of the GameController sits at a PC outside the playing area. He or she will signal any change in the game state to the robots via the wireless as they are announced by the head referee. He or she will also inform the assistant referees when a timed penalty is over and a robot has to be placed back on the field. The operator may also be responsible for time keeping if the head referee has delegated this task.

\subsection{Selection of the Referees}

At least in the preliminaries, the games will be refereed by members of teams from a different Round-Robin group. Each team has to referee a number of these games. For each of the games, it can either provide the head referee and the operator of the GameController, or the two assistant referees. These persons must have good knowledge of the rules as applied in the tournament, and the operator of the GameController must be experienced in using that software. The persons should be selected among the more senior members of a team, and preferably have prior experience with games in the RoboCup Standard Platform (formerly Four Legged) league.

Referees for playoff games will need to be certified (\ie deemed fit to referee) by at least half the teams in the playoffs. Unless they have no eligible referees, each team in the playoffs shall supply at least one referee for the playoffs. For a particular game, each of the teams playing shall be able to veto one and only one eligible referee with no reason required.

\subsection{Referees During the Match}

The head referee and the assistant referees should wear \emph{black clothing and black socks/shoes} and avoid reserved colors for the ball, the goals, and player markings in their clothing. They may enter the field in particular situations, \eg, to remove a robot when applying a penalty. They should avoid interfering with the robots as much as possible.

\section{Changes From 2010}

This is a brief list of this year's changes to the rules.

\begin{itemize}
\item The number of robots per team changed from 3 to 4 (see Sec.~\ref{sec:field_players}).
\item The rules for autonomous and manual positioning changed according to the increased number of robots (see Sec.~\ref{sec:kick-off}).
\item The pushing rules changed substantially, both concerning the pushing between opposing robots as well as team mates (see Sec.~\ref{sec:player_pushing}).
\item Robots are now removed for the rest of the half after the 4th, 6th and 8th pushing (see Sec.~\ref{sec:pushing_results}).
\item The obstruction rules has been removed (formerly Sec. 4.14 in the 2010 rules).

\end{itemize}


\section{Questions/Comments}

Questions or comments on these rules should be mailed to spl\_tech@tzi.de

\end{document}



