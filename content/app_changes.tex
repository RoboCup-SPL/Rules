\section{Changes From 2019}
This is a brief list of rule changes from 2019 to 2022.

\begin{itemize}
  \item General housekeeping and tidy-up of rules.
  \item Remove some duplicated items
\end{itemize}

\subsection*{Field Layout}
\begin{itemize}
  \item Increased size of the penalty box (\cf Section~\ref{sec:field_dim})
  \item Added Goalbox - as same size of the old penalty box (\cf Section~\ref{sec:field_dim})
\end{itemize}

\subsection*{Field Colors}
\begin{itemize}
  \item Removed redundant figure (\cf Section~\ref{sec:field_colors})
\end{itemize}

\subsection*{Robot Players: Hardware}
\begin{itemize}
  \item Added black Nao plating as permitted color~(\cf Section~\ref{sec:hardware})
  \item Permit tape on the battery pack~(\cf Section~\ref{sec:hardware})
\end{itemize}

\subsection*{Robot Players: Team Markers}
\begin{itemize}
  \item Teams must specify which front and back halves of jerseys are combined.~(\cf Section~\ref{sec:hardware})
\end{itemize}

\subsection*{Structure of the Game}
\begin{itemize}
  \item Specify whistle sounds for commencing and ending each half (\cf Section~\ref{sec:game_struct})
\end{itemize}

\subsection*{Robot States}
\begin{itemize}
  \item Updated states for penalty kick (\cf Section~\ref{sec:robot_states})
\end{itemize}

\subsection*{Goal}
\begin{itemize}
  \item Referee signals a goal being scored by a whistle sound~(\cf Section~\ref{sec:goal})
  \item GameController transmission of a goal being scored is delayed by \GoalScoredDelay~(\cf Section~\ref{sec:goal})
\end{itemize}

\subsection*{Indirect Kick}
\begin{itemize}
  \item Rule introduced (\cf Section~\ref{sec:goal})
\end{itemize}

\subsection*{Kick-off}
\begin{itemize}
  \item Specify location of referee whistle
\end{itemize}

\subsubsection*{Kick-off Shot}
\begin{itemize}
  \item Kick-off shot penalty removed.
  \item Kick-off Rule subsumed by the Indirect Kick Rule (\cf Section~\ref{sec:goal}), \& Invalid Goal Rule (\cf Section~\ref{sec:goal})
\end{itemize}

\subsection*{Free Kick}
\begin{itemize}
  \item Change referee phrase for ending the free kick to ``Ball Free'' for consistency with kick-off and for less confusion on  head referees during the game.(\cf Section~\ref{sec:free_kick})
\end{itemize}

\subsubsection*{Goal Kick}
\begin{itemize}
  \item Position of goal kick adjusted to the corner of the goalbox (\cf Section~\ref{sec:kick_in})
\end{itemize}

\subsubsection*{Penalty Kick}
\begin{itemize}
  \item Rule introduced~(\cf Section~\ref{sec:free_kick}).
  \item Penalty Kick procedure introduced~(\cf Section~\ref{sec:penalty_free_kick}).
\end{itemize}

\subsection*{Penalty Kick Shoot-out}
\begin{itemize}
  \item Time for Penalty Kick reduced to \PenaltyKickTime.
  \item Striker robot is placed on the edge of the penalty box~(\cf Section~\ref{sec:penalty_kick})
  \item Goalkeeper robot must remain on the goal line and on its feet~(\cf Section~\ref{sec:penalty_kick})
  \item Removed requirement for timing sudden-death penalty shots as this is no-longer used as a tie-breaker
  \item General Rules housekeeping~(\cf Section~\ref{sec:penalty_shoot-out} \& Section~\ref{sec:sudden_death_shoot_out})
\end{itemize}

\subsection*{Illegal Defender \& Position}
\begin{itemize}
  \item Illegal Defender subsumed into to Illegal Position (\cf Section~\ref{sec:illegal_positioning})
  \item Illegal Position penalties now \textit{follow} the standard removal penalty and increase in penalty time per-robot. The only exception are illegally positioned robots during the set state.
  \item Penalty box illegal positioning applies to robots from both teams (\cf Section~\ref{sec:illegal_positioning})
  \item Number of robots permitted inside the penalty box increased to 3 (\cf Section~\ref{sec:illegal_positioning})
  \item Illegal Defender subsumed into to Illegal Position (\cf Section~\ref{sec:illegal_positioning})
  \item Penalty box illegal positioning applies to robots from both teams (\cf Section~\ref{sec:illegal_positioning})
\end{itemize}

\subsection*{Leaving the Field Penalty}
\begin{itemize}
  \item Clarified leaving the field also applies to walking into the goal posts~(\cf Section~\ref{sec:leaving_field})
\end{itemize}

\subsection*{Jamming}
\begin{itemize}
  \item Explicitly define interference of whistle sounds~(\cf Section~\ref{sec:jamming})
\end{itemize}

\subsection*{Judgement - Head Referee}
\begin{itemize}
  \item Modified head referee's use of the whistle as it is now important for robots to detect the whistle during gameplay. Whistle's are only used for kick-off, end-of-half and goals.
  \item Removed whistle sound for game stuck.
  \item Added requirement to use calls as defined in the rules.
  \item Clarified head referee's role in handling robots and the ball.
  \item Removed requirement for timing sudden-death penalty shots as this is no-longer used as a tie-breaker
\end{itemize}

\subsection*{Official Rules - Qualification Procedure}
\begin{itemize}
  \item Revised qualification requirements for novel contributions and reuse of existing software~(\cf Section~\ref{sec:qualification_procedure_codeuse})
  \item \textbf{ALL TEAMS}: Note novel contribution requirements now apply to all teams.
\end{itemize}


\subsection*{Official Rules - Disqualification during competition}
\begin{itemize}
  \item Added possibility to disqualify a team during competition for serious ethical breaches, or violation of the terms of their qualification~(\cf Section~\ref{sec:disqualification_during_comp})
\end{itemize}

\subsection*{Mixed Team Tournament}
\begin{itemize}
  \item Removed
\end{itemize}

\subsection*{Technical Challenges}
\begin{itemize}
  \item Penalty Shoot Out Challenge added, as per 2018 rules.
\end{itemize}

\subsection*{Field Technical Drawing}
\begin{itemize}
  \item Added.
  \item Contains explicit dimensions accounting for the width of the tape.
\end{itemize}

\subsection*{Wireless Communications}
\begin{itemize}
  \item Introduced limits to the ammount of data sent during robot-to-robot communication.
\end{itemize}

\section{Potential 2022 Changes}
\begin{itemize}
  \item Move position of the penalty spot subject to testing at the 2022 regional workshops and competitions.
\end{itemize}

\section{Alternative Section~\ref{sec:wireless}}
The following is an alternative suggestion for introducing limits to robot-to-robot communication.

Advantages:
\begin{itemize}
  \item More flexible with penalty for violating network limits
  \item More confidence that packages counted to network limits are correct
\end{itemize}

Disadvantages:
\begin{itemize}
  \item More work for teams to implement
  \item Test setups without GameController become harder for teams to set up
\end{itemize}

The TC is looking for feedback whether this alternative is preferrable to the new limits introduced in section \ref{sec:wireless}. Please send your feedback to \url{rc-spl-tc@lists.robocup.org}.

The only wireless hardware allowed to be used by the teams are the wireless network cards built into the NAOs, and the access points provided by the event organizers. All other wireless hardware must be deactivated. A team may be disqualified if one of the team members violates this rule.

Each team will get a range of IP addresses that can be used both for their robots and their computers. The network configuration (\eg IP addresses, channels, SSIDs, and required encryption) of the fields will be announced at the competition site.

Robots are only allowed to use wireless communications to connect to the access points provided by the event organizers, using a static IP address from a range provided by the organizers. Robots may only communicate on fields that are not running an official game or fields which they are playing on.

Robots may use the wireless network only for the following three purposes in exactly the ways described here:
\begin{enumerate}
  \item Communication with the GameController:
  The GameController periodically broadcasts the structure \texttt{RoboCupGameControlData} (representing the current game state) via UDP on the port \texttt{GAMECONTROLLER\_DATA\_PORT}. \texttt{RoboCupGameControlData} and \texttt{GAMECONTROLLER\_DATA\_PORT} are defined in the C header file \texttt{RoboCupGameControlData.h} which is included in the source distribution of the GameController. Robots must periodically indicate their status by sending the structure \texttt{RoboCupGameControlReturnData} on port \texttt{GAMECONTROLLER\_RETURN\_PORT} (also defined in \texttt{RoboCupGameControlData.h}). These packets must be addressed directly to the GameController host; the GameController does not accept broadcast packets.

  \item Communication within the team:
  Team messages are sent as UDP packets on a team-specific port directly to the GameController host. The GameController host will then re-broadcasts packages via UDP on the same port to the other robots of the same team. They can have a payload size (\ie size excluding UDP, IP and link layer headers) of up to \qty{\TeamMessageSize}{\byte} and contain arbitrary data. Conventionally, the team-specific port is 10000 plus the team's number in the GameController. Robots are not allowed to broadcast team messages or address them to any other host than the one running the GameController.

  The number of messages that a team may send during a game is limited to \TeamMessageLimit. The GameController tracks the number of messages that have already been sent and includes the counters per team in \texttt{RoboCupGameControlData}. After a team has reached its limit, the GameController does not forward any messages of that team for the remainder of the game.

\item Debug Communications:
  In order to monitor and visualize the status of the robots on an external computer, robots may send debug messages. Debug messages are UDP packets on a team-specific port conforming to the \texttt{SPLStandardMessage} structure (which also defines their maximum size) from the C header file \texttt{SPLStandardMessage.h}. They must not be sent more often than once per second. Conventionally, the team-specific port is 11000 plus the team's number in the GameController.\todo{Would it be better to use 10000 for this and 11000 for team messages instead for compatibility reasons? Should we enforce unicast?}

  Debug messages must not be received by any other robot, and neither must an external computer send any messages to the robots.
\end{enumerate}

In summary, the following is an exhaustive list of the ways robots may use the wireless network during games:
\begin{itemize}
  \item receive UDP packets on \texttt{GAMECONTROLLER\_DATA\_PORT}
  \item send UDP packets conforming to the \texttt{RoboCupGameControlReturnData} structure on \texttt{GAMECONTROLLER\_RETURN\_PORT} to the GameController host (\ie the one from which they are receiving GameController messages)
  \item receive UDP packets on their own team-specific team communication port
  \item send UDP packets of up to \qty{\TeamMessageSize}{\byte} payload size on their own team-specific team communication port to the GameController host
  \item send UDP packets conforming to the \texttt{SPLStandardMessage} structure on their own team-specific debug communication port up to once every second
\end{itemize}

The use of remote processing/sensing is prohibited.
