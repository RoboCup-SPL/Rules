\section{Technical Challenges}

\subsection{7 vs. 7}
    This competition extends the ideas from the mixed team competition, and 1 vs. 1 remote challenge from RoboCup 2021 to a standardized 7 vs. 7 on-site competition. Another goal is to enforce more collaborative game play.

    \subsubsection{Condition for participation} % PG
        \begin{itemize}
            \item number of own robots
            \item number of pool robots
            \item V6
            \item calibration in given time
        \end{itemize}

    \subsubsection{Rules}
        This challenge bases on all rules from the 5 vs. 5 competition (Sections~\ref{sec:setup_environment} — \ref{sec:judgment}). The following list contains extended and changed rules for this challenge:

        \begin{itemize}
            \item Goali dive % AM
            \item Own pushing % AM
            \item Hardware related penalty % AM
            \item More field players + numbers, starting positions, own and pool robots, v6 % AM
            \item setup procedure % PG
            \item robot pool (3, more if more robots are available) and evaluation % AM
            \item Modus % PG
        \end{itemize}


\subsection{Visual Referee}

    Just an idea for starting with this challenge.
    
    \subsubsection{Challenge Goal}

        For the moment robots receive the decision made by the head referee either from a short whistle or a GC message. To improve the perception of the robot and to listen more to the head referee this challenge introduces visual referee signs.

    \subsubsection{Challenge setup}

        One robot of the challenged team is placed in the center circle facing the head referee standing on the opposite T-junction to the GC. The referee wears red glows. The robot is listing and watching the referee. The general procedure is as follows:

        \begin{itemize}
            \item The referee blows the whistle.
            \item The referee shows his decision for \qty{15}{\second}.
            \item During that time or in the additional \qty{10}{\second} the robot has time to phrase the referee's decision, e.g., ```Kick-off left team'''. \todo{Or we use a TCP message}
        \end{itemize}

        This procedure will be continued for in total four times. Each time, the referee chooses a new decision.

    \subsubsection{Available Decisions}
        
        For each decision (not all, because some decisions have to be shown on place, e.g., throw in) will be described and pictured.

        \begin{itemize}
            \item \textbf{Kick-Off} 
            \begin{figure}
                \includegraphics{figs/kick-off_referee.jpg}
            \end{figure}
        \end{itemize}

    \subsubsection{Challenge evaluation}
        The time from the whistle until the robots starts messaging will be counted for each run.
        For each decision two points can be awarded: One for the right decision itself and one for the right team awarded to or against.
        The ranking is based on the sum of points. Higher number of points leads to a higher ranking. For teams with equal points the sum of the runs will be used. Less time used leads to a higher ranking.
        

\subsection{Dynamic Ball handling}

    This challenge extends the idea of RoboCup 2021's Passing Challenge. The purpose of this challenge is to enhance skills in ball passing and handling, and in robot's movement estimation.

    \subsubsection{Challenge Goal}

    Score as attacking team a goal after a double pass without letting the opponents players touch the ball.

    \subsubsection{Challenge Setup}

    This challenge uses a standard SPL field, with GameController and 3 attacking robots provided by the challenged team and three defending robots operating with a provided common image from another team. A common image will be provided by the community with the standardized setup procedure and with automatic calibration. Every team can propose such an image \todo{deadline}. The image will be tested if they match the requirements. \todo{Was wird getestet} If more than one exists, than multiple images will be provided and for each run a new one will be randomly selected.
    \todo{Criterias for images, how to setup}

    \todo{Drawing of setup and variance}

    The robots are placed by the referees with some randomness as follows:
    \begin{itemize}
        \item \textbf{Attacker:} 1st goal box front line; 2nd next to center line left of center circle; 3rd next to center line right of center circle
        \item \textbf{Defender:} 1st within center circle; 2nd front line penalty box; 3rd goalkeeper in the middle between the two goal posts.
        \item \textbf{Ball:} On penalty spot of the attacking team's side 
    \end{itemize}

    Each team has three attempts to run this challenge.

    \subsubsection{Challenge Execution}

    All six robots have to be in the Wi-Fi. In initial the robots get placed at their randomized starting positions. GC goes from ready directly into set. The ball gets placed, and the head referee starts the challenge execution with one whistle, like at kick-off. If a robot does not listen to the whistle, it will get the delayed playing signal from the GC.

    In playing the following happens: The 1st attacker plays the ball towards the 2nd or 3rd attacker will he is under attack by the 1st defender. A pass counts as a valid pass if the ball stops in front of the receiving robot towards the opponent's goal. The ball has to stop in the vicinity of the receiving robot (at max \qty{1}{\metre} distance between receiver and ball). \todo{Visualization}
    The 1st defender does not walk back in its own half. The objective of the defending team is to intercept the passes, see stopping criteria.
    \todo{3rd defender, remains within goal box}
    \todo{Make clear, attacking defending robot, defending defing robot, goalie}
    \todo{Limit max speed of Defenders}

    After the 2nd or 3rd attacker received the ball, and the ball is in the defender's half, it gets attacked by the 2nd defender. The task of the 2nd or 3rd attacker is now to pass towards the 3rd or 2nd attacking robot, which than shots onto the goal.

    \subsubsection{Challenge Scoring}

    The challenge execution will be stopped, when one of the following events occur:

    \begin{itemize}
        \item A defender touches the ball.
        \item Ball leaves the field.
        \item An attacker pushes.
        \item The execution duration exceeds \qty{3}{\minute}.
        \item A goal scored with zero or one pass.
    \end{itemize}

    A score for the run will be calculated based on the following rules:

    The time in seconds in playing counts. Passing a ball in the back of a receiver adds 10 extra seconds each time. Scoring a goal after two passes subtracts 30 seconds.
    If the ball left within the goal box the field, or it got intercepted by goalkeeper (both after 2 passes) the time counts.
    If the execution gets stopped, the for the run a time of 180 seconds is taken.

    The final result is the mean time out of three runs.

\subsection{Video analysis / statistics}
