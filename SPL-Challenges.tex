% !TeX spellcheck = en_US
\documentclass[12pt]{article}

\usepackage{times,fullpage,xspace,fancyhdr,url,color}
\usepackage[pdftex]{graphicx}
\usepackage[pdftex,
            colorlinks=true,
            urlcolor=black,
            linkcolor=black,
            citecolor=black,
            bookmarksopen=false,
            bookmarksnumbered=true,
            pdfstartview=FitH]{hyperref}

\pdfcompresslevel=9
\newcommand{\leaguename}{RoboCup Standard Platform League (NAO) }
\hypersetup{
 pdftitle={\leaguename Technical Challenges},
 pdfauthor={Technical Committee SPL},
}
\usepackage{microtype}
\usepackage[utf8]{inputenc}
\usepackage{amsmath}
\usepackage{xargs}
\usepackage[colorinlistoftodos,prependcaption,textsize=tiny]{todonotes}
\usepackage{siunitx}
\usepackage[capitalize,noabbrev]{cleveref}
\usepackage[official]{eurosym}
\usepackage[useregional]{datetime2}
\usepackage{subcaption}
\usepackage{enumitem}
\usepackage{xcolor}
\DTMlangsetup[en-GB]{ord=raise,monthyearsep={,\space}}

\newcommandx{\unsure}[2][1=]{\todo[linecolor=red,backgroundcolor=red!25,bordercolor=red,#1]{#2}}
\newcommandx{\change}[2][1=]{\todo[linecolor=blue,backgroundcolor=blue!25,bordercolor=blue,#1]{#2}}
\newcommandx{\info}[2][1=]{\todo[linecolor=green,backgroundcolor=green!25,bordercolor=green,#1]{#2}}
\newcommandx{\improvement}[2][1=]{\todo[linecolor=Plum,backgroundcolor=Plum!25,bordercolor=Plum,#1]{#2}}

% comment 'disable' in to disable all the todo notes :)
\usepackage
[
%disable
]{todonotes}

\usepackage[theorems]{tcolorbox}
\newtcbtheorem[number within=section]{hintbox}{}%
{colback=red!10,colframe=red!45!black,fonttitle=\bfseries}{th}

% !TeX root = ../SPL-Rules.tex
% !TeX spellcheck = en_US
\newcommand{\TotalWidth}{7.4}
\newcommand{\TotalLength}{10.4}
\newcommand{\GoalScoredDelay}{15}
\newcommand{\KickOffAutoTime}{45}
\newcommand{\KickOffBallFreeTime}{10}
\newcommand{\FreeKickTime}{30}
\newcommand{\FreeKickRadius}{0.75}
\newcommand{\VisualSignalTime}{2}
\newcommand{\ReadyDelayTimeChampion}{45}
\newcommand{\ReadyDelayTimeChallenge}{40}
\newcommand{\PlayingDelayTime}{15}
\newcommand{\PenaltyKickTime}{30}
\newcommand{\PenaltyKickSetupTime}{30}
\newcommand{\PenaltyShootoutKickTime}{30}
\newcommand{\StandardPenaltyTime}{45}
\newcommand{\StandardPenaltyIncrease}{10}
\newcommand{\NovelContributionTime}{3 years\xspace}
\newcommand{\GameStuckTime}{30}
\newcommand{\TeamMessageSize}{128}
\newcommand{\TeamMessageLimit}{1200} % Limit of number of packets available to one team during a game with two halves of 10 minutes.
\newcommand{\TeamMessageLimitMinute}{60} % Limit for the average number of packets available to one team during a minute of gameplay.
\newcommand{\MaxJerseyNumber}{20} % the highest allowed jersey number to wear by robot players

% !TeX root = ../SPL-Rules.tex
% !TeX spellcheck = en_US
\newcommand{\LastRCYear}{2022\xspace}
\newcommand{\RCYear}{2023\xspace}
\newcommand{\JerseyApproveSubmissionDate}{2023-05-01}
\newcommand{\CodeReleaseAnnouncementDate}{2023-10-15}


\sloppy
\newcommand{\ie}{\mbox{i.\,e.}\xspace}
\newcommand{\eg}{\mbox{e.\,g.}\xspace}
%\newcommand{\cf}{\mbox{cf.}\xspace}
\newcommand{\cf}{see\xspace}
% \newcommand{\comment}[1]{\marginpar{\pdfannot width 4in height .5in depth 8pt {/Subtype /Text /Contents (#1)}}}
\newcommand{\inparagraph}[1]{\paragraph{#1\hspace{-1em} }}


% some colors
\definecolor{orange}{rgb}{1,0.5,0}
\definecolor{red}{rgb}{1,0,0}
\definecolor{green}{rgb}{0,1,0}


\title{\leaguename\\Technical Challenges}
\author{RoboCup Technical Committee}
\date{(\RCYear technical challenges, as of \today)}

\setlength{\parindent}{0pt}
\setlength{\parskip}{12pt plus 6pt minus 3 pt}
\setcounter{tocdepth}{1}
\widowpenalty=10000
\clubpenalty=10000

\pagestyle{fancy}
\lhead{}
\chead{}
\rhead{}
\lfoot{}
\cfoot{}
\rfoot{}

\renewcommand{\headrulewidth}{0.4pt}
\renewcommand{\footrulewidth}{0.4pt}

% needed to align an image and text correctly side by side
\newcommand{\imagebox}[1]{\raisebox{2ex}{\raisebox{-\height}{#1}}}

\begin{document}

\maketitle

\begin{center}
Questions or comments on the technical challenge rules should be submitted via \url{https://github.com/RoboCup-SPL/Rules/issues}, to the \texttt{\#rule-book} channel on the SPL Discord server, or by mail to \url{rc-spl-tc@lists.robocup.org}.
\end{center}

\newpage

\tableofcontents
\setcounter{tocdepth}{3}

\thispagestyle{fancy}

\clearpage

\cfoot{\thepage}
\setcounter{page}{1}

\section{Introduction}
At RoboCup \RCYear, the Standard Platform League will hold one technical challenge, which is described in this document.
RoboCup \RCYear awards a trophy for winning this challenge and the option for pre-qualification if a team is not pre-qualified by other means.

Technical challenges are used in the SPL to develop technical capabilities which will be used in upcoming RoboCups in the main competition. The purpose is to give teams time to develop solutions and exchange ideas before they will be introduced into the main competition. Challenges are designed to move the league in a direction of further improvement of soccer skills and towards the overall goal of 2050. Each team is strongly encouraged to participate in these challenges to contribute to the league's advancement.

\subsection{Code Publication}
Every team participating in a challenge must publish the corresponding code used in that competition according to Appendix A.7 of the SPL rule book, unless a specific challenge states otherwise.

% !TeX root = ../SPL-Challenges.tex
% !TeX spellcheck = en_US
\section{KICKin\textquotesingle \& Rollin\textquotesingle Challenge}

The objective of this challenge is for a robot to successfully kick a rolling ball from a ramp into a designated goal area. 
Each participating team will record three attempt sets, with a significant time between them.
A more specific schedule will be communicated closer to the competition.
An attempt sets consist of three kick attempts. 

\subsection{Field Setup}

The challenge will take place on a standard SPL field with its official marking. 
The ramp will be positioned perpendicularly adjacent to the halfway line, and aligned with a virtual line extending from the extremity of the center circle on the opponent's side.
The general idea is to position the Nao facing the goal, with the ball moving from its left to its right, or vice versa.
The ramp can be positioned at varying distances from the robots to modify the difficulty.
However, a minimum distance of 30 cm between the ramp and the virtual line must be maintained to ensure the ball rolls directly onto the field.
The participating robot can be positioned at the team’s discretion along the virtual line between the center circle and the penalty mark.
The robot must have one foot on each side of this virtual line.

\subsubsection{Ramp}

The ramp design will be available on the SPL website.
The ramp may have an adjustable angle to modify the difficulty.


\subsection{Challenge Procedure}

At the scheduled time, every participating must regroup on the designated field. 
The ramp will be set up by the technical committee and will remain the same for all attempts executed this day.
Each team will perform their three kicks successively.
Teams may bring as many robots as needed, but they will not be allowed to deploy code between attempts.

\subsection{Scoring}

Each kick will receive a score based on the location of the ball.
These points are non-cumulative: 

\begin{itemize}
	\item 0 points for a missed kick.
	\item 1 point for a deflected ball.
	A ball is considered deflected if it changes direction due to contact with the robot's feet.
	\item 2 points for kicking the ball.
	A ball is considered kicked if it makes contact with the front sensor of the robot's feet.
	\item 3 points for the ball entering the penalty area during the attempt. 
	If the ball leaves the area before coming to a complete stop, without qualifying for subsequent points, this still counts. 
	\item 5 points for the ball entering the goal. 
	\item 100 points for the ball going over the goal between the virtual planes of the two goalposts. 
  \end{itemize}

  A ball can only be considered kicked or deflected if it result from a direct action by the robot.
  Contact resulting from a stationary feet or a backward step will result in 0 points.

  \subsection{Ranking}

  Each team will be ranked based on the average of their best kick from each attempt.
% !TeX root = ../SPL-Challenges.tex
% !TeX spellcheck = en_US
\section{Open Research Challenge}
The idea of this challenge is to create a platform within the Standard Platform League for teams to showcase innovations that contribute to the growth, development, and improvement of the league, beyond the core competition objectives. These innovations aim to enrich the league's ecosystem by fostering collaboration, technological advancement, and accessibility.

This challenge incentivizes teams to focus on league-wide growth areas such as software, hardware, and community alignment. It provides an opportunity for teams to make impactful contributions in areas like fostering innovation through creative solutions, strengthening collaboration within the SPL community, enhancing accessibility for new and existing teams, boosting engagement with spectators and sponsors, and supporting the long-term growth and sustainability of the league.

\subsection{Submission Requirements}
Teams must submit a brief description of their planned contribution by \textbf{July 1, 2025} to the Technical Committee (\url{rc-spl-tc@lists.robocup.org}).
Multiple contributions per team are allowed.
The description should include:
    \begin{itemize}
        \item \textbf{Objective}: The purpose and scope of the innovation.
        \item \textbf{Expected Impact}: How it benefits SPL or improves existing systems.
        \item \textbf{Implementation Feasibility}: Resources and time needed to accomplish/execute the contribution.
    \end{itemize}

\subsection{Presentation Format}
Contributions will be presented through a \textbf{10-minute presentation} during the SPL competition.
Presentations must include:
\begin{itemize}
    \item Clear explanation of the contribution and its impact.
    \item A \textbf{live demonstration}, if feasible, or a \textbf{video demonstration} otherwise. If neither a live nor video demonstration is possible, the team must request an exception by notifying the Technical Committee and explaining the issue.
\end{itemize}
Presentations in the PDF format must be submitted to the Technical Committee on the day of the presentation and will also be uploaded to the \textbf{SPL website} for public review and reference.

\subsection{Full Code Release Requirement}
A \textbf{full code release} is mandatory for participating in this challenge. The code must be publicly available no later than the \textbf{main code release deadline} from the SPL rule book and must be announced over the \url{robocup-nao@lists.robocup.org} mailing list.

Ideally, the code should be released before the competition to allow interested teams and individuals to review and provide feedback!



\subsection{Evaluation Process}
Contributions will be evaluated during the competition using the \textbf{Borda count method\footnote{\url{https://en.wikipedia.org/wiki/Borda_count}}}, to ensure a fair and democratic assessment. Each team leader of the participating teams ranks all the presented contributions except their own. The rankings are aggregated to determine the overall score for each team. Equal overall scores are allowed. In such cases, tied contributions will share the same position in the overall ranking.\\
If a team submits multiple distinct contributions, each will be evaluated and ranked separately, potentially resulting in multiple individual rankings for that team.

\subsubsection*{Judging Criteria:}
\begin{itemize}
    \item \textbf{Relevance}: Alignment with SPL’s goals of growth and improvement.
    \item \textbf{Feasibility}: Practicality of implementation within league resources.
    \item \textbf{Impact}: Potential benefit to the league or teams.
    \item \textbf{Innovation}: Originality and creativity of the idea.
    \item \textbf{Live Demonstration}: Preference should be given to contributions that include a live demonstration.
    \item \textbf{Code Release}: Consideration of whether the code was made available beforehand for review and that it is well documented.
    %\item \textbf{Presentation Quality}: The clarity, engagement, and effectiveness with which the contribution and its impact are communicated.
\end{itemize}

Depending on the number of contributions, it may be necessary to conduct multiple Borda count evaluations. Each evaluation can focus on different sets of judging criteria, allowing every contribution to be assessed under the most appropriate context and ensuring a more nuanced and fair comparison.

\subsection{Scope of Contributions}
Projects should focus on innovations beneficial to the league but not necessarily related to winning matches. Examples include:
\begin{itemize}
    \item \textbf{Common Simulator}: A platform to test robot software binaries in a shared virtual environment.
    \item \textbf{Universal Log File Format}: Standardizing data recording formats for easier analysis and collaboration.
    \item \textbf{Interoperable Code Standards}: Aligning software frameworks across teams and leagues for better portability and sharing.
    \item \textbf{Shared Software Architecture}: Creating modular interfaces to simplify the exchange of robot components between teams.
    \item \textbf{Semi-Automatic Refereeing}: Utilizing field-side cameras and AI for referee assistance.
    \item \textbf{Real-Time Game Statistics}: Broadcasting match data through automated systems and providing teams with these statistics.
\end{itemize}

\subsection{Contributions Archive}
All contributions will be showcased on the \textbf{SPL website}, including:
\begin{itemize}
    \item The presentations from the competition.
    \item Full code releases for each contribution.
\end{itemize}
This ensures that every innovation is documented, celebrated, and accessible for future reference and use by the SPL community.


\end{document}
