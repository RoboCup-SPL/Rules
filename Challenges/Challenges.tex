\documentclass{article}
\usepackage{graphicx}
\usepackage{url}
\usepackage{verbatim}
\usepackage{times}
\usepackage{url}
\usepackage{epsfig}
\usepackage{amsmath}

\begin{document}

\title{Technical Challenges for the RoboCup 2013 Standard Platform League Competition}

\author{RoboCup SPL Technical Committee}

\maketitle

\section{Introduction}
\label{sec:introduction}

There are three technical challenges that will be held at the RoboCup 
2013 Standard Platform League Competition. These are:

\begin{itemize}
\item The Open Challenge (Section~\ref{sec:open})
\item The Passing Challenge (Section~\ref{sec:passing})
\item The Dribbling Challenge (Section~\ref{sec:dribbling})
\item The Drop-In Player Challenge (Section~\ref{sec:dropIn})
\end{itemize}

The team with the top score in a challenge will receive 25 points, each 
position thereafter will receive 1 less point; i.e. 1st = 25pts, 2nd = 24pts, 
3rd = 23pts ... 25th = 1pts. In the case of a draw, each team will receive 
the average of the points allocated to these positions; e.g. if three teams 
tie for 2nd, they will receive $(24+23+22)/3 = 23$ points. Teams not competing 
in a challenge will receive 0 points, also if a team competes but fails to 
score a point (or receive a vote) they will receive 0 points again. The team 
with the highest total score after all challenges is deemed the overall 
challenge winner.

All challenges will use the 2013 field and the 2013 rules will apply.

The challenges will be performed separately in three two hour time slots. 
In each time slot, five minutes are reserved for each team, \emph{three minutes} 
of which will be used for the actual challenge. The remaining two minutes are 
reserved for setup and intermediate stoppages if the challenge requires them.

Ten minutes before each Technical Challenge two hour time slot starts, the teams 
have to provide the robots participating in the challenge to the Technical Committee 
(switched off).

Before each challenge, the robot(s) will be booted and put into the \emph{penalized} 
state. For the start of the challenge they will be unpenalized, either by the 
GameController or manually by pushing the chest button. The GameController will 
configure the robots to team color \emph{blue}.


\section{The Open Challenge}
\label{sec:open}
\newcommand{\openMinNum}{three}

This challenge is designed to encourage creativity within the Standard 
Platform League, allowing teams to demonstrate interesting research in 
the field of autonomous systems. Each team will be given \openMinNum{} 
minutes of time on the RoboCup field to demonstrate their research. 
Each team \emph{should} also distribute a short, one page description of 
their research prior to the competitions. The winner will be decided by 
a vote among the entrants. In particular:

\begin{itemize}
\item 
Teams must describe the content of their demonstration to the technical 
committee at least \emph{four weeks} before the competitions. 
\item 
The demonstration should be strongly related to the scope of the league. 
Irrelevant demonstrations, such as dancing and debugging tool presentations, 
are discouraged.
\item 
Each team may use any number of Aldebaran Nao robots. Teams must arrange
for their own robots.
\item 
Teams have \openMinNum{} minutes to demonstrate their research. This
includes any time used for initial setup. Any demonstration deemed
likely to require excessive time may be disallowed by the technical
committee.
\item 
Teams may use extra objects on the field as part of their
demonstration. \emph{Robots other than the Naos may not be used}.
\item 
The demonstration must \emph{not} mark or damage the field. Any
demonstration deemed likely to mark or damage the field may be
disallowed by the technical committee.
\item
The demonstration may \emph{not} modify the Nao robots.
\item 
The demonstration may use off-board sensors or
actuators, as long as the Nao is still the focus of the challenge.
\item 
The demonstration may use off-board computing power connected over the
wireless LAN. This is the only challenge in which off-board
computation is allowed.
\item 
The demonstration may use off-board human-computer interfaces. This
is the only challenge in which off-board interfaces, apart from the
Game Controller, are allowed.
\end{itemize}

The winner will be decided by a vote among the entrants using a Borda
count (\url{http://en.wikipedia.org/wiki/Borda_count}). Each participating 
team will vote for their top 10 teams in order (excluding themselves).
Teams are encouraged to evaluate the performance based on the
following criteria: technical strength, novelty, expected impact and
relevance to RoboCup. At a time decided by the designated referee,
within 30 minutes of the last demonstration if not otherwise
specified, the captain of each team will submit his or her team's rankings 
by filling out an online form.  Any points awarded by a team to itself will be 
disregarded. The points awarded by the teams are summed and the team with the 
highest total score shall be the winner.

\section{The Passing Challenge}
\label{sec:passing}
This challenge is intended to encourage teams to develop passing skills. In this challenge,
each team will be required to provide three robots, all robots must be in the same
colored uniform (the decision on the color of the uniforms can be made by each team).
Each robot will placed on the field inside a circle of radius 35cm (Figure 1). The center of the circles will be no closer
then 80cm and no further then 200cm apart. The triangle formed by the circles will not
be equilateral, i.e. the distances between robots will be different.

The location of the circles will be made available on the morning of the challenge.
It will be the responsibility of each team to make sure they have set the correct points.
Initially each robot will be placed inside a circle and in the `set' state for 15 seconds,
this will enable them to localise. The robots will then be placed into `playing' and given
two minutes to pass the ball around.

A pass will be regarded as successful when:
\begin{itemize}
\item The passing robot releases the ball from inside its circle \emph{and}
\item the catching robot stops/controls the ball inside its circle. Stops/control will be
left to the referees discretion. Examples are :
\begin{itemize}
\item The ball comes to a complete stop.
\item The robot is capable of hitting the ball from one circle to another without the need for stopping the ball.
\end{itemize}
\end{itemize}

A pass will be deemed \emph{partially} successful if:
\begin{itemize}
\item The passing robot releases the ball from inside its circle \emph{and}
\item the catching robot touches the ball inside the circle but the ball then travels outside the circle.
\end{itemize}

A pass is deemed unsuccessful if:
\begin{itemize}
\item Either robot makes contact with the ball when the ball is outside a circle \emph{or}
\item the ball exits the field.
\end{itemize}

A robot is deemed to be inside a circle if part of one foot is inside the circle. The ball is
inside the circle if some part of the ball is inside the circle. The line is regarded as inside the circle.

Robots may pass between each other in any order, but will be rewarded only for passing
to a different robot then that which passed to it. Scoring of the challenge will be as
follows:

\textbf{3 pts} For a successful non `return' pass that directly follows a successful pass reception.
\textbf{1 pt} For a successful pass.
\textbf{0.5 pt} For a partially successful pass.

If two teams score the same number of points, the result is a draw. All normal game rules apply in the challenge, 
except that if the ball leaves the field, it will be replaced back in the closest circle.
If a rule is violated then any pass resulting from this violation will receive no points.

\section{The Dribbling Challenge}
\label{sec:dribbling}
Dribbling challenge, based on a kids soccer drill (see the attached
pdf).  In this challenge, teams would need to dribble around
cones/beacons/robots from a start point to an end point.  The team
with the best time would win, and penalties would be assessed to teams
for each failure to travel on the correct side of the
cones/beacons/robots.

\section{The Drop-In Player Challenge}
\label{sec:dropIn}

\end{document}


