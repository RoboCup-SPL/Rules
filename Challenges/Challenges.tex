\documentclass[12pt]{article}

\usepackage{times,fullpage,xspace,fancyhdr,url}
\usepackage[pdftex]{graphicx}
\usepackage[pdftex,
            a4paper,
            colorlinks=true,
            urlcolor=black,
            linkcolor=black,
            citecolor=black,
            bookmarksopen=false,
            bookmarksnumbered=true,
            pdfstartview=FitH]{hyperref}

\usepackage{graphicx}
\usepackage{xspace,color}
\pdfcompresslevel=9
\newcommand{\leaguename}{RoboCup Standard Platform League (NAO) }
\hypersetup{
 pdftitle={\leaguename Technical Challenge Proposals 2014},
 pdfauthor={Technical Committee},
}
\usepackage[latin1]{inputenc}
\usepackage{amsmath}
\usepackage{times}

% comment 'disable' in to disable all the todo notes :)
\usepackage
[
%disable
]{todonotes}

\sloppy
\newcommand{\ie}{\mbox{i.\,e.}\xspace}
\newcommand{\eg}{\mbox{e.\,g.}\xspace}
\newcommand{\cf}{\mbox{cf.}\xspace}
\newcommand{\comment}[1]{\marginpar{\pdfannot width 4in height .5in depth 8pt {/Subtype /Text /Contents (#1)}}}
\newcommand{\inparagraph}[1]{\paragraph{#1\hspace{-1em} }}


% some colors
\definecolor{orange}{rgb}{1,0.5,0}
\definecolor{red}{rgb}{1,0,0}
\definecolor{green}{rgb}{0,1,0}


\title{\leaguename \\ Technical Challenge Proposals}
\author{RoboCup Technical Committee}
\date{(2014 rules, as of \today)}

\setlength{\parindent}{0pt}
\setlength{\parskip}{6pt plus 6pt minus 3 pt}
\setcounter{tocdepth}{1}
\widowpenalty=10000
\clubpenalty=10000

\pagestyle{fancy}
\lhead{}
\chead{}
\rhead{}
\lfoot{}
\cfoot{}
\rfoot{}

\renewcommand{\headrulewidth}{0.4pt}
\renewcommand{\footrulewidth}{0.4pt}

% needed to align an image and text correctly side by side
\newcommand{\imagebox}[1]{\raisebox{2ex}{\raisebox{-\height}{#1}}}



\begin{document}

\maketitle

For RoboCup 2014, the Technical Committee identified five possible technical challenges, which are described in this document. Out of these five, two or three challenges will be selected based on the vote of all SPL teams. For this purpose, an online poll is open at \todo{Add URL} XXX until \todo{Add Deadline} YYY.

The proposals described in this document have to be considered as drafts. The final challenge rules might be slightly different and will contain more details.

Questions or comments on these rules should be mailed to {\small \url{rc-spl-tc@lists.robocup.org}}.

\vfill

\renewcommand\contentsname{Challenges}
\tableofcontents
\setcounter{tocdepth}{1}

\thispagestyle{fancy}

\clearpage

\cfoot{\thepage}
\setcounter{page}{1}



% % % % % % % % % % % % % % % % % % % % % % % %



\section{Open Challenge}
\newcommand{\openMinNum}{three}

This challenge is designed to encourage creativity within the Standard 
Platform League, allowing teams to demonstrate interesting research in 
the field of autonomous systems. Each team will be given \openMinNum{} 
minutes of time on the RoboCup field to demonstrate their research.

Each team that wishes to compete in this challenge \emph{must} send a 
short, one page document describing their demonstration to the technical 
committee by \textbf{May 27, 2013}.  Teams who do not submit this description by 
the deadline will not be allowed to compete in this challenge. These 
descriptions will be posted on the website before the competition.

The winner will be decided by a vote among all the SPL teams. In particular:

\begin{itemize}
\item 
The demonstration should be strongly related to the scope of the league. 
Irrelevant demonstrations, such as dancing and debugging tool presentations, 
are discouraged.
\item 
Each team may use any number of Aldebaran NAO robots. Teams must arrange
for their own robots.
\item 
Teams have \openMinNum{} minutes to demonstrate their research. At most one 
additional minute may be used for initial setup. Any demonstration deemed
likely to require excessive time may be disallowed by the technical
committee.
\item 
Teams may use extra objects on the field as part of their
demonstration. \emph{Robots other than the NAOs may not be used}.
\item 
The demonstration must \emph{not} mark or damage the field. Any
demonstration deemed likely to mark or damage the field may be
disallowed by the technical committee.
\item
The demonstration may \emph{not} modify the NAO robots.
\item 
The demonstration may use off-board sensors or actuators, as long 
as the NAO is still the focus of the challenge.  This is the only 
challenge in which off-board sensors or actuators are allowed.
\item 
The demonstration may use off-board computing power connected over the
wireless LAN. This is the only challenge in which off-board
computation is allowed.
\item 
The demonstration may use off-board human-computer interfaces. This
is the only challenge in which off-board interfaces, apart from the
Game Controller, are allowed.
\end{itemize}

The winner will be decided by a vote among the SPL teams using a Borda
count (\url{http://en.wikipedia.org/wiki/Borda_count}). Each SPL 
team will vote for their top 10 teams in order (excluding themselves).
Teams are encouraged to evaluate the performance based on the
following criteria: technical strength, novelty, expected impact and
relevance to RoboCup. At a time decided by the designated referee,
within one hour of the last demonstration if not otherwise
specified, the captain of each team will submit his or her team's rankings 
by filling out an online form at \url{http://goo.gl/vIVTr}.  Any points 
awarded by a team to itself will be disregarded. The points awarded by the 
teams will be summed, and the team with the highest total score shall be the winner.

\newpage



% % % % % % % % % % % % % % % % % % % % % % % %



\section{Walking Challenge}
Bla!

\newpage



% % % % % % % % % % % % % % % % % % % % % % % %



\section{Any Place Challenge}
Bla!

\newpage



% % % % % % % % % % % % % % % % % % % % % % % %



\section{Autonomous Refereeing Challenge}

Dietmar has to add a description here.

\newpage


% % % % % % % % % % % % % % % % % % % % % % % %



\section{Sound Recognition Challenge}

Dietmar has to add a description here.



\end{document}

