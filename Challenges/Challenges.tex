\documentclass{article}
\usepackage{graphicx}
\usepackage{url}
\usepackage{verbatim}
\usepackage{times}
\usepackage{url}
\usepackage{epsfig}
\usepackage{amsmath}

\begin{document}

\title{Technical Challenges for the RoboCup 2013 Standard Platform League Competition}

\author{RoboCup SPL Technical Committee}

\maketitle

\section{Introduction}
\label{sec:introduction}

There are three technical challenges that will be held at the RoboCup 
2013 Standard Platform League Competition. These are:

\begin{itemize}
\item The Open Challenge (Section~\ref{sec:open})
\item The Passing Challenge (Section~\ref{sec:passing})
\item The Dribbling Challenge (Section~\ref{sec:dribbling})
\item The Drop-In Player Challenge (Section~\ref{sec:dropIn})
\end{itemize}

The team with the top score in a challenge will receive 25 points, each 
position thereafter will receive 1 less point; i.e. 1st = 25pts, 2nd = 24pts, 
3rd = 23pts ... 25th = 1pts. In the case of a draw, each team will receive 
the average of the points allocated to these positions; e.g. if three teams 
tie for 2nd, they will receive $(24+23+22)/3 = 23$ points. Teams not competing 
in a challenge will receive 0 points, also if a team competes but fails to 
score a point (or receive a vote) they will receive 0 points again. The team 
with the highest total score after all challenges is deemed the overall 
challenge winner.

All challenges will use the 2013 field and the 2013 rules will apply.

The challenges will be performed separately in three two hour time slots. 
In each time slot, five minutes are reserved for each team, \emph{three minutes} 
of which will be used for the actual challenge. The remaining two minutes are 
reserved for setup and intermediate stoppages if the challenge requires them.

Ten minutes before each Technical Challenge two hour time slot starts, the teams 
have to provide the robots participating in the challenge to the Technical Committee 
(switched off).

Before each challenge, the robot(s) will be booted and put into the \emph{penalized} 
state. For the start of the challenge they will be unpenalized, either by the 
GameController or manually by pushing the chest button. The GameController will 
configure the robots to team color \emph{blue}.


\section{The Open Challenge}
\label{sec:open}
\newcommand{\openMinNum}{three}

This challenge is designed to encourage creativity within the Standard 
Platform League, allowing teams to demonstrate interesting research in 
the field of autonomous systems. Each team will be given \openMinNum{} 
minutes of time on the RoboCup field to demonstrate their research. 
Each team \emph{should} also distribute a short, one page description of 
their research prior to the competitions. The winner will be decided by 
a vote among the entrants. In particular:

\begin{itemize}
\item 
Teams must describe the content of their demonstration to the technical 
committee at least \emph{four weeks} before the competitions. 
\item 
The demonstration should be strongly related to the scope of the league. 
Irrelevant demonstrations, such as dancing and debugging tool presentations, 
are discouraged.
\item 
Each team may use any number of Aldebaran Nao robots. Teams must arrange
for their own robots.
\item 
Teams have \openMinNum{} minutes to demonstrate their research. This
includes any time used for initial setup. Any demonstration deemed
likely to require excessive time may be disallowed by the technical
committee.
\item 
Teams may use extra objects on the field as part of their
demonstration. \emph{Robots other than the Naos may not be used}.
\item 
The demonstration must \emph{not} mark or damage the field. Any
demonstration deemed likely to mark or damage the field may be
disallowed by the technical committee.
\item 
The demonstration may \emph{not} use any off-board sensors or
actuators, or modify the Nao robots.
\item 
The demonstration may use off-board computing power connected over the
wireless LAN. This is the only challenge in which off-board
computation is allowed.
\item 
The demonstration may use off-board human-computer interfaces. This
is the only challenge in which off-board interfaces, apart from the
Game Controller, are allowed.
\end{itemize}

The winner will be decided by a vote among the entrants using a Borda
count (\url{http://en.wikipedia.org/wiki/Borda_count}). Each participating 
team will vote for their top 10 teams in order (excluding themselves).
Teams are encouraged to evaluate the performance based on the
following criteria: technical strength, novelty, expected impact and
relevance to RoboCup. At a time decided by the designated referee,
within 30 minutes of the last demonstration if not otherwise
specified, the captain of each team will submit his or her team's rankings 
by filling out an online form.  Any points awarded by a team to itself will be 
disregarded. The points awarded by the teams are summed and the team with the 
highest total score shall be the winner.

\section{The Passing Challenge}
\label{sec:passing}

\section{The Dribbling Challenge}
\label{sec:dribbling}

\section{The Drop-In Player Challenge}
\label{sec:dropIn}

\end{document}


