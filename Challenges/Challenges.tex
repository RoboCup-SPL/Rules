\documentclass[12pt]{article}

\usepackage{times,fullpage,xspace,fancyhdr,url}
\usepackage[pdftex]{graphicx}
\usepackage[pdftex,
            a4paper,
            colorlinks=true,
            urlcolor=black,
            linkcolor=black,
            citecolor=black,
            bookmarksopen=false,
            bookmarksnumbered=true,
            pdfstartview=FitH]{hyperref}

\usepackage{graphicx}
\usepackage{xspace,color}
\pdfcompresslevel=9
\newcommand{\leaguename}{RoboCup Standard Platform League (NAO) }
\hypersetup{
 pdftitle={\leaguename Technical Challenges 2016},
 pdfauthor={Technical Committee},
}
\usepackage[latin1]{inputenc}
\usepackage{amsmath}
\usepackage{times}

% comment 'disable' in to disable all the todo notes :)
\usepackage
[
%disable
]{todonotes}

\sloppy
\newcommand{\ie}{\mbox{i.\,e.}\xspace}
\newcommand{\eg}{\mbox{e.\,g.}\xspace}
\newcommand{\cf}{\mbox{cf.}\xspace}
\newcommand{\comment}[1]{\marginpar{\pdfannot width 4in height .5in depth 8pt {/Subtype /Text /Contents (#1)}}}
\newcommand{\inparagraph}[1]{\paragraph{#1\hspace{-1em} }}

\long\def\commentk#1{{\bf ++K: #1++}}

% some colors
\definecolor{orange}{rgb}{1,0.5,0}
\definecolor{red}{rgb}{1,0,0}
\definecolor{green}{rgb}{0,1,0}


\title{\leaguename \\ Technical Challenges}
\author{RoboCup Technical Committee}
\date{(2016 technical challenge rules, as of \today)}

\setlength{\parindent}{0pt}
\setlength{\parskip}{6pt plus 6pt minus 3 pt}
\setcounter{tocdepth}{1}
\widowpenalty=10000
\clubpenalty=10000

\pagestyle{fancy}
\lhead{}
\chead{}
\rhead{}
\lfoot{}
\cfoot{}
\rfoot{}

\renewcommand{\headrulewidth}{0.4pt}
\renewcommand{\footrulewidth}{0.4pt}

% needed to align an image and text correctly side by side
\newcommand{\imagebox}[1]{\raisebox{2ex}{\raisebox{-\height}{#1}}}



\begin{document}

\maketitle

At RoboCup 2016, the Standard Platform League will hold two different technical challenges, which are described in this document.

The scores earned in each challenge will vary in magnitude.  Hence, they must be scaled before calculating the overall technical challenge rankings.  Teams who do not participate in a challenge will receive 0 points for that challenge.  The team with the highest total score for a challenge will get 25 points for that challenge, while the team with the lowest total score for a challenge will get 5 points for that challenge.  A linear equation will then be fit to these two points, and each other participating team in that challenge will gain points for that challenge based on this equation.

For both challenges, no changes of code or configuration are allowed for any participating team after the first team starts the challenge. 

Questions or comments on these rules should be mailed to {\small \url{rc-spl-tc@lists.robocup.org}}.

\vfill

\renewcommand\contentsname{Challenges}
\tableofcontents
\setcounter{tocdepth}{1}

\thispagestyle{fancy}

\clearpage

\cfoot{\thepage}
\setcounter{page}{1}

\newcommand{\openMinNum}{three}


% % % % % % % % % % % % % % % % % % % % % % % %


\section{No Wi-Fi Challenge}

The purpose of this challenge is to foster communication without the use of wireless networks.

\subsection{Setup}
TODO

\subsection{Procedure}
TODO

\subsection{Score}
TODO

\newpage
% % % % % % % % % % % % % % % % % % % % % % % %



\section{Outdoor Challenge}

\subsection{Purpose}

The purpose of the challenge is to foster skills for playing in natural lighting on a more realistic playing surface.

\subsection{Setup}
TODO

\subsection{Procedure}
TODO 

\subsection{Score}
TODO

\end{document}

