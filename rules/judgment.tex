% !TeX root = ../SPL-Rules.tex
% !TeX spellcheck = en_US
\section{Judgment}
\label{sec:judgment}

The referees are the only persons permitted on the carpeted area (\ie the field and the border area).

\subsection{Head Referee}
\label{sec:head_referee}

The head referee is in charge of the game.
Any decision of the head referee is valid.
The head referee's decision is final and can not be changed afterwards, even by video proof.
There is no discussion about decisions during the game, neither between the assistant referees and the head referee, nor between the audience or the teams and the head referee.

The head referee announces decisions by a clear loud call, and (as required) whistle sound.
The whistle, or where there is no whistle the first verbal word of the referees calls, defines the point in time at which the decision is made.
The referees should make efforts to use consistent and clear calls, and it is preferable for referees to use the calls as specified in these rules.\footnote{
  The calls specified in these rules are detailed in English.
  With the agreement of the teams, the referees may use suitable calls in any language.
  The exception to this are technical challenges that depend on the calls as specified.
}
The intention of specifying the referee calls is for clarity and consistency across games.

Where a whistle is required, the head referee first whistles and then announces the reason for the whistle.
The head referee may choose to use any normal sports whistle.
Each whistle sound should be short and not too loud as to interfere with other fields and simultaneous games.
The head referee must \textit{only} sound the whistle in circumstances described in these rules.
There are three circumstances when the whistle is sounded: kick-off~(\cf \cref{sec:kick-off}), a goal~(\cf \cref{sec:goal}), and ending a half of gameplay~(\cf \cref{sec:game_struct}).

The head referee should avoid handling the ball (except for placing a ball for kick-off), and avoid handling the robots.
Their duty is to monitor and adjudicate the game.
The head referee should only handle robots and the ball if absolutely necessary to expedite gameplay or removal of penalized robots, where the assistant referees are otherwise occupied or too far away.

The head referee should be equipped with a suitable referee jersey, whistle, coin, and black or dark-blue socks.

The head referee may decide at any point before or during a game to relocate any objects around the field, or direct persons to another position around the field.

\subsection{Assistant Referees}
\label{sec:assist_referee}

The assistant referees handle the robots and the ball.
They take the robots out when they are penalized, and they put the robots in again.
If a team requests to pick up a robot, an assistant referee will pick it up and give it to one of the team members once the head referee approves.
An assistant referee will also put the robot back on the field.
An assistant referee will also replace the ball when it goes off the field or becomes stuck between a players feet.
At the discretion of the head referee, more tasks can be delegated to the assistants (\cref{sec:referee_delegation}).

Assistant referees should only enter the field to execute a decision made by the main referee.
They should not prevent robots from falling during the game.

A game has at least two assistant referees. If agreed upon by the referee teams
or under certain circumstances, additional assistants may be present,
up to a total maximum of four. See \cref{sec:refSelection}.

The assistant referees should be equipped with a suitable referee jersey, and black or dark-blue socks.

\subsubsection{The Standing Assistant}

Up to one assistant referee may be designated as the \textit{standing assistant}.
This is especially encouraged if the game includes additional assistants
beyond the standard two.

If a \textit{standing assistant} is present, they are tasked with standing
on the T-junction opposite to the technical area at all times and perform
the gestures for the beginning of the \texttt{ready} phase and for set plays
instead of the head referee.

They may be delegated other tasks by the head referee: in particular, it is strongly
encouraged that they keep track of the indirect kick rule, given its relation
to set plays.

\subsection{GameController Operator}
\label{sec:gameControllerOp}

The operator of the GameController sits at a PC in the technical area.
As with the head referee, the operator should make efforts to use consistent and clear calls.
They will signal any change in the game state to the robots via the wireless as they are announced by the head referee.
Note that for both kick-offs and goals, the moment of whistling is determining, not the verbal announcement of the head referee.
The operator will also inform the assistant referees when a timed penalty is over and a robot has to be placed back on the field.
They should announce events that occur automatically in the GameController due to elapsed time, such as the ball coming into play after a kick-off, penalty kick or free kick, or the state changing from \texttt{ready} to \texttt{set}.
They are also responsible for keeping the time of each half.
They should count aloud the remaining seconds in a half once the time remaining is \qty{5}{\second} or less.
Finally, they should repeat the calls of the head referee to make sure it was heard correctly.

\subsection{Pre-game referee meeting and task delegation}
\label{sec:referee_delegation}

Before the game starts, the people scheduled to serve as referees meet up to discuss the
upcoming game. At least, they must decide which team is going to provide the head
referee and the GameController operator and which is going to provide the assistants,
and whether the head referee is going to delegate any duties to the assistants.
Other topics that ensure a smooth cooperation among the referees can also be discussed.

The head referee should talk to the assistants to determine
what tasks or lesser decisions, if any, they wish to delegate to them
to ensure that the game is arbitrated as smoothly as possible.
This is left to the discretion of the head referee, based on their expertise in the role
and their ability to focus on multiple events happening in the game at the same time
and apply the corresponding rules.

The head referee must clearly communicate what tasks are delegated
to which person, so that everyone understands their duties during the game.

If no agreement can be found, the default is that the responsibility for most
calls and decisions falls upon the head referee, as determined by the rules.

Common examples of tasks that can be delegated are:

\begin{itemize}
  \item Keeping track of the number of robots from the same team that deliberately played the ball
        since the last free kick for the purposes of the indirect kick rule (\cref{sec:indirect_kick_default})
        and communicating to the head referee whether a goal is valid or not.
        The final call is still made by the head referee.
  \item Determining which team is to be given a free kick when the ball goes
        out of the field (\cref{sec:kick_in}) and communicating this to the head referee.
        The final call is still made by the latter.
  \item Physically executing the gestures for the beginning of the \texttt{ready}
        phase and/or for set plays at the T-junction opposite to the technical area
        (\cref{sec:free_kick_gesture}) instead of the head referee.
  \item Providing help for ``Leaving the Field'' penalties (\cref{sec:leaving_field}). In a lesser form,
        this could mean that assistants are allowed to pick up a robot that is
        clearly intending to leave the carpeted area or going to get damaged
        even if the head referee has not called ``Leaving the Field'' on it,
        holding it until they get the attention of the head referee
        and the call is made.
        Alternatively, the assitant referees may be given full power to make
        ``Leaving the Field'' calls in place of the head referee, but this solution
        also requires the approval of the GameController operator since that means
        they will have multiple people to pay attention to.
  \item Indicating other violations requiring a penalty, with the final call still
        being made by the head referee.
\end{itemize}

The above list is not prescriptive: the head referee can always choose to delegate
zero, one, some, or all of these tasks. It is also not exhaustive: through
discussion among the head referee, the assistants, and the GameController operator,
other tasks not listed here may be identified and delegated.

Care should be taken to not overburden any one person and not to blur the roles
of head and assistant referees. Conflicts in the authority of the referees should
be avoided, but if any do occur, the head referee's decision is final.

\subsection{Referee--Team Communication}
\label{sec:referee_team_communication}

Both teams send a representative called team captain to the field \qty{10}{\minute} before their match starts.
This time should be used to welcome each other, assign team colors (\cf~\cref{sec:team_markers}), choose side and kick-off (\cf~\cref{sec:field_side_and_initial_kickoff}), and discuss match related topics.

During the match only the team captains are allowed to communicate with the head referee.
Only the team captains and two more people per team are allowed to stay next to the game controller tables.
The rest of the team locates themselves around the other sides of the field if they want to watch the match.
This allows the referees easier communication with the team and the game controller operator gets less disturbed.

After the match the teams thank the referees for their duty.

During all phases of the match teams and referees are communicating with respect to each other.

\subsection{Referee List For Friendly Games}
\label{sec:referee_list}

During the RoboCup competition setup days and during the game days, several teams may want to participate in friendly games with each other.
In this case, there should be a list of all the head and assistant referees that are willing to judge a game.
The referees on the list should be voluntary but participation should be encouraged.
This is especially recommended for those who wish to gain referee experience.
Teams are still free to choose their own referees.

The OC should be in charge of maintaining the list and should be approached if one should want to volunteer.

\subsection{Referees During the Match}
\label{sec:referee_during_match}

The head referee and the assistant referees should wear socks \emph{of black or dark blue color} and avoid reserved colors (white and green) in their leg clothing.
They may enter the field in particular situations, \eg, to remove a robot when applying a penalty.
They should avoid interfering with the robots as much as possible.

\subsubsection{Visual signal}
\label{sec:visual-signal}

For visual signal execution, the referee should have bare hands as much as possible.
Exceptions apply as described in the following subsections and for small jewelry.

\paragraph{Challenge Shield}

The referee executing the signal will wear red gloves to increase visibility; however, teams are advised not to rely solely on this visual cue, as gloves may be omitted in future years.